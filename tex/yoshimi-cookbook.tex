%-------------------------------------------------------------------------------
% yoshimi-cookbook
%-------------------------------------------------------------------------------
%
% \file        yoshimi-cookbook.tex
% \library     Documents
% \author      Chris Ahlstrom
% \date        2015-07-12
% \update      2015-07-15
% \version     $Revision$
% \license     $XPC_GPL_LICENSE$
%
%     This document provides LaTeX documentation for yoshimi.  It is similar
%     in structure to the yoshimi-doc project.
%
%-------------------------------------------------------------------------------

\documentclass[
 11pt,
 twoside,
 a4paper,
 headinclude,
 footinclude,
 final                                 % versus draft
]{article}

% This file and its clone in yoshimi-doc should be kept in sync:

\input{yoshimi-docs-structure}         % specifies document structure and layout

\makeindex

\begin{document}

\title{A Yoshimi Cookbook}
\author{Chris Ahlstrom\\
   (\texttt{ahlstromcj@gmail.com})}
\date{\today}
\maketitle
\tableofcontents
\listoffigures                         % print the list of figures
\listoftables                          % print the list of tables

% Change the paragraph style to remove indenting and put a line between each
% paragraph.  This could be moved up into the preamble, but then would
% affect the spacing of the TOC and LOF, LOT noted above.

\setlength{\parindent}{0pt}
\setlength{\parskip}{1ex plus 0.5ex minus 0.2ex}

\section{Introduction}
\label{sec:introduction}

   This document is a follow-on to the author's
   "A Yoshimi User Manual" \cite{yoshimidoc}.
   The user manual attempts complete coverage of the user-interface and
   concepts behind \textsl{Yoshimi}.
   This cookbook attempts to provide recipes to solve some common problems
   in getting \textsl{Yoshimi} to perform at its best for the user.

\subsection{Project Structure}
\label{subsec:introduction_project_structure}

   The "Yoshimi Cookbook" project consists of two parts:

   \begin{itemize}
      \item The source material for this document.
      \item A self contained "yoshimi" configuration and data
            section to support the examples in this cookbook.
   \end{itemize}

   The documentation source-files are provided in the \texttt{tex}
   directory.  They are use to create the cookbook via Makefiles and
   the external "latexmk" project.  The result of a "make" is a new
   PDF of the cookbook in the \texttt{pdf} directory.  The latest
   PDF is always provided there so that one does not have to install the
   external projects needed to create it.

   The configuration, banks, presets, and instrument files can be used
   to supplement or replaces the user's own configuration and data files.

\subsection{What Game Shall We Play Today?}
\label{subsec:introduction_what_game}

   There are a number of recipes that are hinted at in the user manual, but
   that solve problems that the author has encountered while using
   \textsl{Yoshimi}.

   \begin{itemize}
      \item \textbf{Banks and MIDI}.
         \textsl{Yoshimi} has had recent modifications to support
         bank-switching and using program-change messages to make
         \textsl{Yoshimi} a more flexible MIDI playback tool.
      \item \textbf{General MIDI}.
         It should be possible to set up one or more banks that
         are General MIDI compliant.
      \item \textbf{Usage of Modulators}.
         \textsl{Yoshimi} provides a number of modulation setups,
         but it isn't clear how to use them, especially the ring modulator.
      \item \textbf{More! ...}
   \end{itemize}

% Concepts

%-------------------------------------------------------------------------------
% cookbook_concepts
%-------------------------------------------------------------------------------
%
% \file        cookbook_concepts.tex
% \library     Documents
% \author      Chris Ahlstrom
% \date        2015-07-15
% \update      2015-07-15
% \version     $Revision$
% \license     $XPC_GPL_LICENSE$
%
%     Provides the concepts.
%
%-------------------------------------------------------------------------------

\section{Concepts}
\label{sec:concepts}

   This section, like its counterpart in our \textsl{Yoshimi User Manual},
   presents some useful concepts, while keeping them out of the way.

\subsection{Concepts / Terms}
\label{subsec:concepts_terms}

   This section doesn't provide comprehensive coverage of terms.  It
   covers mainly terms that puzzled the author at first.

\subsubsection{Concepts / Terms / cent}
\label{subsubsec:concepts_terms_cent}

   The \textbf{cent}
   \index{cent}
   is a logarithmic unit of measure used for musical
   intervals.  Twelve-tone equal temperament divides the octave into 12
   semitones of 100 cents each. Typically, cents are used to measure
   extremely small finite intervals, or to compare the sizes of comparable
   intervals in different tuning systems.
   The interval of one cent is much too small to be heard between
   successive notes.

   Since the detuning provided in \textsl{Yoshimi} is based primarily on
   cents (and octaves), it pays to understand cents.  If a given frequency
   \texttt{f'} is offset from another frequency \texttt{f}, the
   relationships between them in semitones are:

   \[f' = f * 2^s/12\]

   \[s = 12 log (f'/f) / log 2\]

   In cents, these relationships become:

   \[f' = f * 2^s/1200\]

   \[s = 1200 log (f'/f) / log 2\]

   These relationships hold whether \texttt{f'} is less than or greater than
   \texttt{f}.  They provide an easy way to determine how much to detune a
   frequency in \textsl{Yoshimi}.

%-------------------------------------------------------------------------------
% vim: ts=3 sw=3 et ft=tex
%-------------------------------------------------------------------------------


% Instruments

%-------------------------------------------------------------------------------
% cookbook_instruments
%-------------------------------------------------------------------------------
%
% \file        cookbook_instruments.tex
% \library     Documents
% \author      Chris Ahlstrom
% \date        2015-07-12
% \update      2015-07-16
% \version     $Revision$
% \license     $XPC_GPL_LICENSE$
%
%     Provides the cookbooks_instruments section of yoshimi-cookbook.tex.
%
%-------------------------------------------------------------------------------

\section{Creating Instruments}
\label{sec:cookbook_instruments}

   One of our goals in using \textsl{Yoshimi} is to support
   \textsl{General MIDI} (\textsl{GM})
   to the greatest extent possible.

   However, no banks have been created with GM in mind.  And many of the
   instruments, though given names that indicate what they are intended to
   be, fall well short of being recognizable per their name; they should be
   doable with a complex synthesizer like \textsl{Yoshimi}.
  
   It is true that there are a vast number of patches out there.  The author
   has attempted a survey of them, and the task is all but impossible.
   Still, many candidates have been identified.  Other candidates might be
   suitable with a little tweaking.

   Here are a number of categories of instruments for which we want to
   assemble an improved set of instruments.

   \begin{enumber}
      \item \textbf{Bells}
      \item \textbf{Ethnic}
      \item \textbf{Drums}
      \item \textbf{Effects}
      \item \textbf{Piano}
      \item \textbf{Leads}
      \item \textbf{Guitar}
      \item \textbf{Strings} (individual and ensemble)
      \item \textbf{Bass}
      \item \textbf{Saxophones}
   \end{enumber}

   For these recipes, the \texttt{banks} directories will
   be stored in the following directory of this project:
   \index{directories!demo bank}

   \begin{verbatim}
      yoshimi/banks
      yoshimi/banks/demo
   \end{verbatim}

\subsection{Bells}
\label{subsec:cookbook_instruments_bells}

   The bells patches we've heard so far are nice, but a bit anemic.
   
   \index{ring modulation}
   Good bell patches are easier with ring modulation, done right.  We're not
   sure if there are any such patches extant; please send us to them if
   there are some.

   \index{bells}
   In the meantime, creating bells is a good excuse to master
   \textsl{Yoshimi}'s ring modulator.
   However, we will first learn how to create a reasonable, clangy bell
   using just a few voices in an ADDsynth part, and no need for modulation.

\subsubsection{Bells by Voices}
\label{subsec:cookbook_instruments_bells_by_voices}

   The following table comes from a tutorial (\cite{bellsimple}).  Along
   with a spectrum shown in reference \cite{bellspectrum}, it allows us to
   recreate a simple, but realistic bell.
   \index{bells!spectrum}
   In this table, \texttt{F} represents the fundamental frequency, i.e. the
   note being played.

\label{table:simple_bell_tones}
\begin{longtable}{c c c c}
   \caption{Simple Bell Tones} \\
   \hline
      \textbf{Wave Number} &
      \textbf{Frequency} &
      \textbf{Cents Offset} &
      \textbf{Relative Amplitude} \\
   \hline
   \endfirsthead

   1 &  0.56F &  -1000 &   0.5   \\
   2 &  0.92F &  -140  &   1.0   \\
   3 &  1.19F &  +300  &   0.5   \\
   4 &  1.71F &  +930  &   0.25  \\
   5 &  2.00F &  +1200 &   0.125 \\
   6 &  2.74F &  +1745 &   0.125 \\
   7 &  3.00F &  +1901 &   0.125 \\
   8 &  3.76F &  +2290 &   0.125 \\
   9 &  4.00F &  +2400 &         \\
\end{longtable}

   Note that the frequencies are relative to the fundamental frequency (F).
   Also note that wave 2 (close to F) can be missing, and the sound still is
   bell-like.

   \index{files!bells addsynth}
   The file \texttt{yoshimi/banks/demo/Bells-simple-addsynth.xiz} is the
   result of following the steps below.  We start, as usual, with a
   newly-started \textsl{Yoshimi} instance.

   \begin{enumber}
      \item Open the ADDsynth editing window by clicking the \textbf{Edit}
         button in the bottom panel, and then clicking the ADDsynth
         \textbf{Edit} button in the edit window.
      \item Click on the \textbf{Show Voice Parameters} button.
         Note that it is \textbf{Current Voice 1}, and it should be enabled.
      \item For voice 1, make the following settings:
         \begin{enumber}
            \item \textbf{Octave}: Set to 0.
            \item \textbf{Detune Type}: Set to E1200cents.
            \item \textbf{FREQUENCY Detune}: Set to -1000 approximately.
         \end{enumber}
      \item Go to voice 2, and make the following settings:
         \begin{enumber}
            \item \textbf{Octave}: Set to 0.
            \item \textbf{Detune Type}: Set to E1200cents.
            \item \textbf{FREQUENCY Detune}: Set to -140 approximately.
         \end{enumber}
      \item Go to voice 3, and make the following settings:
         \begin{enumber}
            \item \textbf{Octave}: Set to 0.
            \item \textbf{Detune Type}: Set to E1200cents.
            \item \textbf{FREQUENCY Detune}: Set to 300 approximately.
         \end{enumber}
      \item Go to voice 4, and make the following settings:
         \begin{enumber}
            \item \textbf{Octave}: Set to 0.
            \item \textbf{Detune Type}: Set to E1200cents.
            \item \textbf{FREQUENCY Detune}: Set to 930 approximately.
         \end{enumber}
      \item Go to voice 5, and make the following settings:
         \begin{enumber}
            \item \textbf{Octave}: Set to 1.
            \item \textbf{Detune Type}: Set to Default.
            \item \textbf{FREQUENCY Detune}: Set to 0.
         \end{enumber}
      \item Go to voice 6, and make the following settings:
         \begin{enumber}
            \item \textbf{Octave}: Set to 1.
            \item \textbf{Detune Type}: Set to E1200cents.
            \item \textbf{FREQUENCY Detune}: Set to 545 approximately.
         \end{enumber}
      \item Go to voice 7, and make the following settings:
         \begin{enumber}
            \item \textbf{Octave}: Set to 1.
            \item \textbf{Detune Type}: Set to E1200cents.
            \item \textbf{FREQUENCY Detune}: Set to 700 approximately.
         \end{enumber}
      \item Go to voice 8, and make the following settings:
         \begin{enumber}
            \item \textbf{Octave}: Set to 1.
            \item \textbf{Detune Type}: Set to E1200cents.
            \item \textbf{FREQUENCY Detune}: Set to 1090 approximately.
         \end{enumber}
   \end{enumber}

   These settings then end up roughly matching the settings of the first 8
   waves in \tableref{table:simple_bell_tones}.
   \index{bells}
   This instrument isn't perfect.  It's not quite equally tempered, though
   close.  The character of the tone changes a bit as the notes get higher.
   One can fiddle with the relative amplitudes of the various voices to
   change the character of this sound.

\subsubsection{Ring Modulation with 440 Hz Tone}
\label{subsec:cookbook_instruments_ring_mod_440}

   Now for an initial demonstration of ring modulation.
   This demonstration does not quite create a bell tone, but does show
   the sound of modulation.

   Start with a fresh \textsl{Yoshimi} and a cleared instrument ("Simple
   Sound").  Open the virtual keyboard using the \textbf{virKbd} button.
   Click a key and verify that you can hear a tone.
   \index{"C" note}
   \index{comma key}
   We'll use the middle C
   key (the "comma" on the PC keyboard) as a reference.  We will call it
   the "C" note.

   The following steps will set up two tones, voice 1 and voice 2, and voice
   2 will use voice 1 as an external modulator.
   Note that you can accomplish most of these steps by loading the project
   file
   \index{bells}
   \index{files!bells 440}
   \texttt{yoshimi/banks/demo/Bells-440-ring-modulation.xiz}, but use
   that only as a last resort.

   \begin{enumber}
      \item Open the ADDsynth editing window by clicking the \textbf{Edit} button
         in the bottom panel, and then clicking the ADDsynth \textbf{Edit}
         button in the edit window.
      \item In the \textbf{Amplitude Env} sub-panel, increase the
         \textbf{D.dt} and \textbf{R.dt} to give the current
         sound a nice slow decay.
      \item Click on the \textbf{Show Voice Parameters} button.
         Note that it is \textbf{Current Voice 1}, and it should be enabled.
      \item Switch to \textbf{Current Voice} number 2 and enable it.
         Play the "C" note, and observe that it is the same frequency, but
         louder.
      \item Move the \textbf{FREQUENCY Detune} slider a bit, and play the "C"
         note.  It should sound the same as before, but change slowly in
         amplitude, as heard and as seen on the \textbf{VU meter}.
         Try to set the detune back to 0; this is easier if you highlight
         the tuning knob and use the left or right arrow keys.
      \item In the \textbf{MODULATOR} section of voice 2, for \textbf{Type},
         select the \textbf{RING} value.  (However, feel free to select one
         of the other modulators, to experiment, once you've mastered
         the ring modulator.)  Press the "C" key again, and notice
         that the tone character changes a bit.  This is due to the internal
         modulator.
      \item For \textbf{External Mod.} for voice 2, select
         \textbf{Ext.M} 1, to use the voice 1 as the internal modulator.
         The "C" note may change in character, but only slightly.
         Apparently the default internal modulator is the same as the
         default external voice 1 waveform.
      \item To actually hear some modulation, we have to separate the
         frequencies of voice 1 and voice 2.  Click the \textbf{440Hz}
         check-box in the \textbf{FREQUENCY} section of voice 1.  Press the
         "C" key and verify hearing a two-tone signal, somewhat like a phone
         tone.
      \item Now go back to voice 2 and click the \textbf{Change} button to
         bring up the ADDsynth oscillator dialog.
      \item Move the slider to maximum for harmonic 10.  Press the "C" key
         and verify the new sound (a bit like a car horn).
         Set the sliders back to 0, and "C" will be a single tone again.
      \item Change the \textbf{Octave} values of voice 2 in its
         \textbf{FREQUENCY} section and listen to the effects.
   \end{enumber}

   Now we need to see if we can apply modulation across instruments.
   Sadly, this does not seem to be possible.

   Increase the \textbf{D.dt} and \textbf{R.dt} values of the main
   \textbf{Amplitude Env} to give this sound the onset and decay of a bell,
   and it then sounds less abstract, and more like a bell.
   Of course, this kind of bell is even less tunable than the simple
   bell of the previous section.

   Another thing to try with this setup is to simply change voice 2 to use
   different types of modulators besides \textbf{RING}.
   \textbf{MORPH} sounds basically identical to \textbf{RING}.
   \textbf{PM} seems to expose higher harmonics, making the sound louder and
   brighter.
   \textbf{FM} sounds similar to PM, but softer and smoother.
   \textbf{PITCH} is disabled.

   Another experiment is to disable the modulator (voice 1 here) and see how
   that changes the sound; all it should do is drop voice 1 from the
   spectrum -- voice 1 will still be used as the modulator.

   Finally, by adding a slow decay to this sound, it becomes amazingly more
   bell-like.

\subsubsection{Complex Bells by Ring Modulation}
\label{subsec:cookbook_instruments_bells_by_ring_mod}

   The next step is to make the bells more complex, by combining the methods
   of the previous two sections.
   Recall \tableref{table:simple_bell_tones}.
   It shows the 9 frequencies in the simple bell spectrum, though we could
   define only 8 of them.  How can we best add extra frequencies?
   \index{bells}
   We can ring-modulate the higher frequencies against one of the lower
   frequencies.

\label{table:ring_mod_bell_tones}
\begin{longtable}{l l l l l}
   \caption{Ring Modulation Bell Tones} \\
   \hline
      \textbf{Wave Number} &
      \textbf{Frequency} &
      \textbf{Mod Frequency} &
      \textbf{f2-f1} &
      \textbf{f2+f1} \\
   \hline
   \endfirsthead

   1 &  0.56F &  ----- &  -----   &  -----   \\
   2 &  0.92F &  0.56F &  0.36F   &  1.48F   \\
   3 &  1.19F &  0.56F &  0.63F*  &  1.75F*  \\
   4 &  1.71F &  0.56F &  1.15F*  &  2.27F   \\
   5 &  2.00F &  0.56F &  1.44F   &  2.56F   \\
   6 &  2.74F &  0.56F &  2.18F   &  3.30F   \\
   7 &  3.00F &  0.56F &  2.44F   &  3.56F   \\
   8 &  3.76F &  0.56F &  3.20F   &  4.32F   \\
\end{longtable}

   The asterisk marks frequencies that are close to existing
   frequencies.   Luckily, there are only three of them, so our
   modulation should add a good number of frequencies.

   \begin{enumber}
      \item Load the file \texttt{yoshimi/banks/demo/Bells-simple-addsynth.xiz}
         to save a lot of steps.  The next steps add voice 1 as a ring
         modulator for voices 2 through 8.
      \item Open the ADDsynth editing window by clicking the \textbf{Edit}
         button in the bottom panel, and then clicking the ADDsynth
         \textbf{Edit} button in the edit window.
      \item Click on the \textbf{Show Voice Parameters} button.
         Note that it is \textbf{Current Voice 1}, and it should be enabled.
      \item Go to voice 2 and do the following steps:
         \begin{enumber}
            \item In the \textbf{MODULATOR} section (greyed out), change the
               \textbf{Type} from \textbf{OFF} to \textbf{RING}.
            \item Changes the \textbf{External Mod.} dropdown from
               \textbf{Off} to \textbf{ExtMod. 1}.
         \end{enumber}
      \item Go to voice 3 and repeat those steps.  Note how all the voices
         below the current voice become available as modulators.
   \end{enumber}

   We saved the result in the file
   \index{files!bells ring mod}
   \texttt{yoshimi/banks/demo/Bells-ringmod-addsynth.xiz} for safe-keeping.

   QUESTION:  If one loads and instrument and tinkers with it, but do not
   save it, does \textsl{Yoshimi} save it on exit anyway?

\subsection{Ethnic}
\label{subsec:cookbook_instruments_ethnic}

   We've found a simple steel drum instruments, but think we might do better,
   creating one using ADDsynth and one using PADsynth.

   Instruments we have not found, and would like to synthesize, are:
   bagpipes and arabic pipes.

\subsubsection{Ethnic / Steel Drums}
\label{subsubsec:cookbook_instruments_ethnic_steeldrums}

   \index{steel drums}
   There is a steel-drum instrument that ships with \textsl{Yoshimi}:
   \texttt{/usr/share/yoshimi/banks/The\_Mysterious\_Bank/0122-pseudo\_steeldrums.xiz}.
   It is an ADDsynth module comprised of three voices:

   \begin{enumber}
      \item A \textbf{Unison}-enabled voice of \textbf{Size} = 10 and a
         \textbf{Frequency Spread} of 44.6 cents.
      \item Another voice that is exactly the same as the first, except that
         it has its \textbf{Amplitude Env} sub-panel enabled, to add more
         volume and character to the instrument, it is stronger on the
         right, and, most importantly, an octave higher.
      \item Another voice that is exactly the same as the second, 
         except it is an octave lower than voice 0.
   \end{enumber}

   If voice 2 and 3 are disabled, the instrument still sounds reminiscent of
   steel drums, so obviously the overall amplitude envelope is important.

   Can we do better?  Well, the instrument above sounds too pristine.
   We should be able to add some "dirt" to the instrument to make it sound
   more lively.

\begin{figure}[H]
   \centering 
   \includegraphics[scale=0.85]{contrib/steel_drum_spectrum.png}
   \caption{Typical Steel Drum Spectrum}
   \label{fig:cookbook_bank_steeldrum_spectrum}
\end{figure}

   Taking a cue from this figure, our steel drums extend the original
   by adding a couple more tones at octave intervals.  Also, some slight
   detuning was introduced to add to the flavor.
   We could probably add a couple more, and carefully contour their
   amplitude levels to match the spectrum levels shown above.

   To hear the ADDsynth steel drum sound, load the file
   \texttt{yoshimi/banks/demo/Add\_Pseudo\_Steel\_Drums.xiz}.

   Not content with that, with our hands behind our back, we pull
   SUBsynth from a hat.  The SUBsynth settings for a steel drum are shown in
   the following figure:

\begin{figure}[H]
   \centering 
   \includegraphics[scale=0.75]{demo/steeldrums-subsynth-editor.png}
   \caption{Steel Drum SUBsynth Configuration}
   \label{fig:cookbook_bank_steeldrum_subsynth}
\end{figure}

   Note the top box of slider controls.  It sets the amplitudes of
   the harmonics, and should vaguely resemble the spectrum diagram.
   The lower box of slider controls sets the bandwidth of each harmonic,
   with the fundamental frequency being very narrow.

   To hear the SUBsynth steel drum sound, load the file
   \texttt{yoshimi/banks/demo/Sub\_Pseudo\_Steel\_Drums.xiz}.

\subsection{Drums}
\label{subsec:cookbook_instruments_drums}

   We want a decent drum kit that attempts to fill in the gaps for a
   GM-compliant drum kit with solid sounds, with
   the help of an existing kit.

   It turns out that a "Natural Drum Kit", which we'd found separately on
   the Internet a long time ago, is now part of the instruments
   installed with \textsl{Yoshimi}.
   \index{drum kit}
   \index{GM!drum kit}
   But long ago we used some of the sounds
   from various kits to create our own "natural drum kit", and extended some
   of the sounds across more (pitched) keys so that any MIDI drum note would
   produce \textsl{some} sound.  We also made sure the sounds were laid out
   in GM format as much as possible.

   Fire up \textsl{Yoshimi} and load the instrument stored in
   \texttt{yoshimi/banks/demo/Natural\_Drum\_Kit\_Basic.xiz}, and we'll walk
   through it.

\begin{figure}[H]
   \centering 
   \includegraphics[scale=0.85]{demo/natural_drum_kit_from_ds_2.png}
   \caption{Natural Drum Kit from DS 2}
   \label{fig:cookbook_bank_natural_drum_kit}
\end{figure}

   \textbf{Item 1} is the master control for the whole kit, determining the
   range of keys that it covers and the effect (if any) it goes through.

   \textbf{Item 2} and \textbf{Item 3} provide the two parts of the
   "natural" snare drum.  Both parts are composed of an ADDsynth and a
   SUBsynth section, and they provide 3 pitches of snare.

   \index{natural drum!snare}
   The "Snare - Stick + Snares" sections....

   \index{natural drum!hihats}
   \index{natural drum!cymbal}
   \index{natural drum!side stick}
   \index{natural drum!toms}
   \index{natural drum!extra toms}
   \index{natural drum!bass drums}
   \index{natural drum!xxxxxxx}
   \index{natural drum!xxxxxxx}
   \index{natural drum!xxxxxxx}
   MORE DESCRIPTIONS
   MORE DESCRIPTIONS
   MORE DESCRIPTIONS
   MORE DESCRIPTIONS
   MORE DESCRIPTIONS

   Now, without defining more than one drum kit, we have only about 15
   "drums" available to us in \textsl{Yoshimi}.  So we filled in the missing
   drums with more "toms", just so some sound will be made.  The frequencies
   of the upper toms get pretty crazy!  Here's a diagram of the keyboard
   layout.  Correlate it with
   \figureref{fig:cookbook_bank_natural_drum_kit}, to understand the
   abbreviations.

\begin{figure}[H]
   \centering 
   \includegraphics[scale=0.75]{demo/natural_drum_keyboard_layout.png}
   \caption{Natural Drum Kit Keyboard Layout}
   \label{fig:cookbook_bank_natural_drum_kit_layout}
\end{figure}

   For your reference, here is the full GM drum layout.  The diagram is
   taken from WikiMedia.org.

\begin{figure}[H]
   \centering 
   \includegraphics[scale=0.75]{demo/GMStandardDrumMap.png}
   \caption{General MIDI Drum Kit Keyboard Layout}
   \label{fig:cookbook_bank_general_drum_kit_layout}
\end{figure}

\subsection{Effects}
\label{subsec:cookbook_instruments_effects}

   This section documents the various "effects" instruments we've created.

\subsubsection{Effects / Dial Tones}
\label{subsubsec:cookbook_instruments_effects_dialtones}

   We've created a nice dial-tone effect that we'll describe here.
   Dial tones consist of two notes, as shown by the
   \textbf{Low F} and \textbf{High F} columns in the following table.

\label{table:effect_dial_tones}
\begin{longtable}{l l l l l l l l l l}
   \caption{DTMF Frequencies Table} \\
   \hline
      \textbf{Tag} &
      \textbf{DTMF} &
      \textbf{Kit\#} &
      \textbf{MIDI\#} &
      \textbf{Low} &
      \textbf{Low F} &
      \textbf{Actual F} &
      \textbf{High} &
      \textbf{High F} &
      \textbf{Actual F} \\
   \hline
   \endfirsthead

   1 &  1        & 5  & 53 & F3 &  697 &  705 &  F5     & 1209 &  1245  \\
   2 &  2        & 6  & 77 & F5 &  697 &  705 &  F5     & 1336 &  1337  \\
   3 &  3        & 7  & 89 & F6 &  697 &  698 &  F5     & 1477 &  1468  \\
   4 &  4        & 8  & 55 & G3 &  770 &  770 &  G5 -   & 1209 &  1236  \\
   5 &  5        & 9  & 79 & G5 &  770 &  776 &  G5 -   & 1336 &  1334  \\
   6 &  6        & 10 & 91 & G6 &  770 &  773 &  G5 -   & 1477 &  1462  \\
   7 &  7        & 11 & 57 & A3 &  852 &  855 &  G\#5 + & 1209 &  1245  \\
   8 &  8        & 12 & 81 & A5 &  852 &  868 &  G\#5 + & 1336 &  1327  \\
   9 &  9        & 13 & 93 & A6 &  852 &  866 &  G\#5 + & 1477 &  1480  \\
   * &  *        & 14 & 59 & B3 &  941 &  948 &  A\#5 + & 1209 &  1257  \\
   0 &  0        & 15 & 83 & B5 &  941 &  968 &  A\#5 + & 1336 &  1281  \\
  \# &  \#       & 16 & 95 & B6 &  941 &  950 &  A\#5 + & 1477 &  1480  \\
   A &  A        & -- & -- & A2 &  697 &  --- &  F5     & 1633 &  ---   \\
   B &  B        & -- & -- & B2 &  770 &  --- &  G5 -   & 1633 &  ---   \\
   C &  C        & -- & -- & C2 &  852 &  --- &  G\#5 + & 1633 &  ---   \\
   D &  D        & -- & -- & D2 &  941 &  --- &  A\#5 + & 1633 &  ---   \\
   b &  busy     & 2  & 71 & B4 &  480 &  472 &  B4 -   & 620  &  622   \\
   r &  ringback & 3  & 69 & A4 &  440 &  440 &  A4     & 480  &  480   \\
   d &  dialtone & 4  & 65 & F4 &  350 &  350 &  F4     & 440  &  440   \\
\end{longtable}

   This table is implemented in a \textsl{Yoshimi kit}.  Each note in the
   kit is created by making an ADDsynth instrument with two voices.  The
   lower voice generally corresponds to the note being play, with an offset,
   if needed, to achieve close to the proper frequency for the lower note of
   the DTMF tone.
   The second voice corresponds to the other note, and it is detune
   appropriately to achieve close to the proper frequency for the upper note
   of the DTMF tone.

   The following figure shows the kit dialog:

\begin{figure}[H]
   \centering 
   \includegraphics[scale=1.0]{demo/DTMF_kit_edit.png}
   \caption{Kit Edit Dialog for DTMF Kit}
   \label{fig:cookbook_instruments_dtmf_kit_edit}
\end{figure}

   To edit the kit, follow the steps below.  If desired, open the
   instrument file
   \index{files!DTMF kit}
   \texttt{yoshimi/banks/demo/DTMF\_Kit.xiz} to save some
   work.

   \begin{enumber}
      \item Open the kit editing window by clicking the \textbf{Edit}
         button in the bottom panel, and then clicking the
         \textbf{Kit Edit} button in the edit window.
      \item Make sure that the \textbf{Mode} is set to \textbf{Single}.
      \item Make sure that the \textbf{Drum mode} is enabled.
      \item For all 16 kit items, make sure that the \textbf{FX.r}
         selections are set to \textbf{OFF}.
      \item For kit items 2 to 16, enable the the \textbf{ADsynth}
         check-box.
      \item For kit items 2 to 16, perform the following procedure to set up
         the two frequencies correctly as per the table above:
         \begin{enumber}
            \item In the kit editor, click the \textbf{Name} field and enter
               the name of the DTMF tone item being edited.
            \item In the kit editor, set \textbf{Min.k} and \textbf{Max.k}
               to the value of the note that is less than or equal to the
               lower note of the item listed in the table.
            \item Click the ADDsynth \textbf{edit} button.
            \item Click on the \textbf{Show Voice Parameters} button.
               Note that it is \textbf{Current Voice 1}, and it should be
               enabled.
            \item Given the frequency for the note being edited, detune
               voice 1 to achieve the desired lower frequency.
            \item Change to voice 2, and enable it.
            \item Given the frequency for the note being edited, detune
               voice 1 to achieve the desired higher frequency.
         \end{enumber}
   \end{enumber}

   The "Actual F" values were verifed using 24-Hz resolution (at 1200 Hz)
   in the spectrum analyzer built into
   \index{Audacity}
   Audacity.
   Sometimes it took a few
   tries to get the best possible frequency.  We could list the detuning
   values in a table; for now, you can see the values we ended up using.

   The following figure shows the layout on the \textsl{Yoshimi} virtual
   keyboard:

\begin{figure}[H]
   \centering 
   \includegraphics[scale=1.0]{demo/DTMF-layout.png}
   \caption{DTMF Layout on the Keyboard}
   \label{fig:cookbook_instruments_DTMF_layout}
\end{figure}

\subsection{Piano}
\label{subsec:cookbook_instruments_piano}

   TODO.

\subsection{Leads}
\label{subsec:cookbook_instruments_leads}

   TODO.

\subsection{Guitar}
\label{subsec:cookbook_instruments_guitar}

   TODO.

\subsection{Strings} (individual and ensemble)
\label{subsec:cookbook_instruments_strings}

   TODO.

\subsection{Bass}
\label{subsec:cookbook_instruments_bass}

   TODO.

\subsection{Saxophones}
\label{subsec:cookbook_instruments_saxophones}

   TODO.

%-------------------------------------------------------------------------------
% vim: ts=3 sw=3 et ft=tex
%-------------------------------------------------------------------------------


% Banks and General MIDI

%-------------------------------------------------------------------------------
% cookbook_banks
%-------------------------------------------------------------------------------
%
% \file        cookbook_banks.tex
% \library     Documents
% \author      Chris Ahlstrom
% \date        2015-07-12
% \update      2015-07-19
% \version     $Revision$
% \license     $XPC_GPL_LICENSE$
%
%     Provides the cookbooks_banks section of yoshimi-cookbook.tex.
%
%-------------------------------------------------------------------------------

\section{Banks and General MIDI}
\label{sec:cookbook_banks}

   Banks are discussed quite heavily in the user manual \cite{yoshimidoc}.
   Banks have evolved quite a bit in \textsl{Yoshimi}, and are
   a powerful way to manage instruments, and more amenable to automation
   than ever.

   In this section, we will attempt to set up a basic bank that is
   compliant with the General MIDI specification.  In order to do so, we
   will cherry pick instruments from the package that is provided when
   \textsl{Yoshimi} is installed, renaming them as needed to fit into the
   appropriate General MIDI slot.

   One problem is the selection of the \textsl{best} instrument for a given
   General MIDI program number.  There are simply too many to be able to
   evaluate them all.

   For this recipe, the \texttt{banks} and \texttt{presets} will
   be stored in the following directories of this project:
   \index{directories!demo bank}
   \index{directories!GM basic bank}
   \index{directories!demo presets}

   \begin{verbatim}
      yoshimi/banks
      yoshimi/banks/demo
      yoshimi/banks/gm-basic
      yoshimi/presets
   \end{verbatim}

\subsection{Creating a Basic GM Bank}
\label{subsec:cookbook_banks_basic_gm}

   Creating even a basic General MIDI bank is beset with issues, even if one
   has at hand a large number of pre-built instruments.

   First, what is the purpose of the General MIDI specific?
   \index{GM}
   \index{General MIDI}
   To provide a
   dependable set of instruments so that tunes will sound basically similar
   on different GM-compliant synthesizers.  That's about it.  It doesn't
   guarantee that the sounds are consistent, nor does it guarantee that they
   are all of high quality.  The "FX", "Lead", and "Pad" instruments provide
   ambiguous descriptions that a wide range of sounds might fit.
   Getting a complete and high-quality set of sounds is extremely difficult.

   Second, evaluating a large number of pre-built instruments takes a lot of
   work.  We'd done some of this work for another project, and never
   finished.  Nor is the naming of such instruments all that helpful; many
   of the file-names are misleading.  Finding decent matches for a GM
   instrument takes time.

   Third, there are many GM instruments for which we've been able to find no
   good \textsl{ZynAddSubFX}/\textsl{Yoshimi} counterpart.  The only options
   are to pick a tolerable match, build a tolerable match oneself, or just
   plug in any old sound and wait for others to step up.

   Nonetheless, let's forge ahead.  The project file
   \texttt{contrib/instrument.ods}
   \index{GM!spreadsheet}
   is a \textsl{LibreOffice} spreadsheet
   that represents some research in finding GM-compatible instruments.
   It's pretty bad; maybe 50\% useful.

   We converted it to \texttt{contrib/gmcopy} to copy the files
   into the project directory \texttt{yoshimi/banks/gm/basic}.
   We show the banks in table~\ref{table:gm_basic_files}
   for convenience.

   \index{GM!table}

%-------------------------------------------------------------------------------
% gm_basic_table
%-------------------------------------------------------------------------------
%
% \file        gm_basic_table.tex
% \library     Documents
% \author      Chris Ahlstrom
% \date        2015-06-15
% \update      2015-07-17
% \version     $Revision$
% \license     $XPC\_GPL\_LICENSE$
%
%     Provides the concepts.
%
%-------------------------------------------------------------------------------

\label{table:gm_basic_files}
\begin{longtable}{|l l|}

\caption{GM Basic Files} \\

\hline
   \textbf{General MIDI Instrument} &
      \textbf{Yoshimi Instrument Used} \\
\hline
\endfirsthead

\hline
   \textbf{General MIDI Instrument} &
      \textbf{Yoshimi Instrument Used} \\
\hline
\endhead

\hline
   Continued next page & \\
\hline
\endfoot

\hline
   End of table & \\
\hline
\endlastfoot

   \textbf{0001-Acoustic Grand Piano} &
      SynthPiano/0033-Analog Piano 1 \\
   \textbf{0002-Bright Acoustic Piano} &
      SynthPiano/0034-Analog Piano 2 \\
   \textbf{0003-Electric Grand Piano} &
      SynthPiano/0143-Space Piano \\
   \textbf{0004-Honky-tonk Piano} &
      SynthPiano/0068-Synth Piano 3 fat \\
   \textbf{0005-Electric Piano 1} &
      Rhodes/0002-DX Rhodes 2 \\
   \textbf{0006-Electric Piano 2} &
      Rhodes/0007-Dig Rhodes \\
   \textbf{0007-Harpsichord} &
      Piano/0139-Home Piano \\
   \textbf{0008-Clavinet} &
      Misc Keys/0060-Clavinet 1 \\
   \textbf{0009-Celesta} &
      Bells/0002-Music\_Box \\
   \textbf{0010-Glockenspiel} &
      Bells/0011-Glass bells \\
   \textbf{0011-Music Box} &
      Bells/0013-Tiny bells \\
   \textbf{0012-Vibraphone} &
      Chromatic Percussion/0045-Vibes no\_trem \\
   \textbf{0013-Marimba} &
      Chromatic Percussion/0056-FM marimba \\
   \textbf{0014-Xylophone} &
      Will\_Godfrey\_Collection/0001-Xylophone \\
   \textbf{0015-Tubular Bells} &
      Chromatic Percussion/0097-Marimba 3 \\
   \textbf{0016-Dulcimer} &
      Plucked/0004-Plucked 4 \\
   \textbf{0017-Drawbar Organ} &
      Organ/0001-Organ 1 \\
   \textbf{0018-Percussive Organ} &
      Organ/0012-Organ 12 \\
   \textbf{0019-Rock Organ} &
      Organ/0068-Square Organ \\
   \textbf{0020-Church Organ} &
      Organ/0061-Great Organ \\
   \textbf{0021-Reed Organ} &
      Reed\_and\_Wind/0039-Reed 7 \\
   \textbf{0022-Accordion} &
      Organ/0097-Accordion Pad 1 \\
   \textbf{0023-Harmonica} &
      Reed\_and\_Wind/0099-Sharp Reed \\
   \textbf{0024-Tango Accordion} &
      Organ/0101-Accordion 1 \\
   \textbf{0025-Acoustic Guitar nylon} &
      Piano/0144-Soft Piano1 \\
   \textbf{0026-Acoustic Guitar steel} &
      Guitar/0065-Clean Guitar1 \\
   \textbf{0027-Electric Guitar jazz} &
      Guitar/0066-Electric Guitar \\
   \textbf{0028-Electric Guitar clean} &
      Guitar/0133-Smooth Guitar \\
   \textbf{0029-Electric Guitar muted} &
      Guitar/0035-Short \\
   \textbf{0030-Overdriven Guitar} &
      Guitar/0042-Trash Guitar 3 \\
   \textbf{0031-Distortion Guitar} &
      Guitar/0005-Dist Guitar 5 \\
   \textbf{0032-Guitar harmonics} &
      Laba170bank/0028-PianoBell \\
   \textbf{0033-Acoustic Bass} &
      Will\_Godfrey\_Collection/0045-Steel Bass \\
   \textbf{0034-Electric Bass finger} &
      Bass/0009-Electric bass 1 \\
   \textbf{0035-Electric Bass pick} &
      Bass/0041-Electric\_Bass \\
   \textbf{0036-Fretless Bass} &
      Bass/0050-Fretless Bass \\
   \textbf{0037-Slap Bass 1} &
      Rhodes/0042-Hard Rhodes1 \\
   \textbf{0038-Slap Bass 2} &
      Rhodes/0042-Hard Rhodes1 \\
   \textbf{0039-Synth Bass 1} &
      Bass/0013-FM rubber bass \\
   \textbf{0040-Synth Bass 2} &
      Bass/0024-Moog bass \\
   \textbf{0041-Violin} &
      Strings/0051-Synth Violin 2 Fat \\
   \textbf{0042-Viola} &
      Strings/0051-Synth Violin 2 Fat \\
   \textbf{0043-Cello} &
      Strings/0051-Synth Violin 2 Fat \\
   \textbf{0044-Contrabass} &
      Bass/0005-Bass 5 \\
   \textbf{0045-Tremolo Strings} &
      Strings/0001-Saw Strings 1 \\
   \textbf{0046-Pizzicato Strings} &
      Strings/0003-Saw Strings 3 \\
   \textbf{0047-Orchestral Harp} &
      Pads/0065-Soft Pad \\
   \textbf{0048-Timpani} &
      Noises/0018-Gun \\
   \textbf{0049-String Ensemble 1} &
      VDX/0065-Strings \\
   \textbf{0050-String Ensemble 2} &
      folderol collection/0029-Full Strings \\
   \textbf{0051-Synth Strings 1} &
      Strings/0010-Strings Pad1 \\
   \textbf{0052-Synth Strings 2} &
      Strings/0014-Strings Pad5 \\
   \textbf{0053-Choir Aahs} &
      Choir\_and\_Voice/0001-AHH Choir 1 \\
   \textbf{0054-Voice Oohs} &
      Choir\_and\_Voice/0004-Voice OOH \\
   \textbf{0055-Synth Voice} &
      Choir\_and\_Voice/0005-Choir Pad1 \\
   \textbf{0056-Orchestra Hit} &
      Misc/0010-Industrial orchestra \\
   \textbf{0057-Trumpet} &
      Leads/0027-Prophet horn 2 \\
   \textbf{0058-Trombone} &
      Brass/0033-Analog Brass 1 \\
   \textbf{0059-Tuba} &
      Brass/0001-FM Thrumpet \\
   \textbf{0060-Muted Trumpet} &
      Synth/0001-Soft Synth 1 \\
   \textbf{0061-French Horn} &
      Brass/0034-Analog Brass 2 \\
   \textbf{0062-Brass Section} &
      Brass/0007-Synth Brass 5 \\
   \textbf{0063-Synth Brass 1} &
      Brass/0003-Synth Brazz 1 \\
   \textbf{0064-Synth Brass 2} &
      Brass/0004-Synth Brazz 2 \\
   \textbf{0065-Soprano Sax} &
      Reed\_and\_Wind/0066-Fat Reed2 \\
   \textbf{0066-Alto Sax} &
      Reed\_and\_Wind/0065-Fat Reed1 \\
   \textbf{0067-Tenor Sax} &
      Reed\_and\_Wind/0037-Reed 5 \\
   \textbf{0068-Baritone Sax} &
      Reed\_and\_Wind/0099-Sharp Reed \\
   \textbf{0069-Oboe} &
      Reed\_and\_Wind/0040-Reed 8 \\
   \textbf{0070-English Horn} &
      Brass/0034-Analog Brass 2 \\
   \textbf{0071-Bassoon} &
      Will\_Godfrey\_Collection/0102-Bassoon \\
   \textbf{0072-Clarinet} &
      Reed\_and\_Wind/0006-Clarinet \\
   \textbf{0073-Piccolo} &
      Will\_Godfrey\_Collection/0071-Ocarina \\
   \textbf{0074-Flute} &
      Will\_Godfrey\_Collection/0057-Soft Flute \\
   \textbf{0075-Recorder} &
      Will\_Godfrey\_Collection/0059-Ocarina \\
   \textbf{0076-Pan Flute} &
      Will\_Godfrey\_Collection/0127-Pan Pipe \\
   \textbf{0077-Blown Bottle} &
      Will\_Godfrey\_Collection/0125-Bottle \\
   \textbf{0078-Shakuhachi} &
      Will\_Godfrey\_Collection/0125-Pan Pipe 32 \\
   \textbf{0079-Whistle} &
      Will\_Godfrey\_Collection/0027-Ghost Whistle \\
   \textbf{0080-Ocarina} &
      Flute/0071-Ocarina \\
   \textbf{0081-Lead 1 square} &
      Leads/0022-Square lead \\
   \textbf{0082-Lead 2 sawtooth} &
      Louigi\_Verona\_Workshop/0008-saw-lead \\
   \textbf{0083-Lead 3 calliope} &
      Leads/0018-Sine lead \\
   \textbf{0084-Lead 4 chiff} &
      chip/0018-Chiffer\_Chip \\
   \textbf{0085-Lead 5 charang} &
      Louigi\_Verona\_Workshop/0001-progressive-lead-1 \\
   \textbf{0086-Lead 6 voice} &
      Choir\_and\_Voice/0067-Vocal Morph 3 \\
   \textbf{0087-Lead 7 fifths} &
      chip/0017-SuperSquare1 \\
   \textbf{0088-Lead 8 bass lead} &
      Strings/0157-Dual Strings Oct2 \\
   \textbf{0089-Pad 1 new age} &
      Pads/0028-Ethereal \\
   \textbf{0090-Pad 2 warm} &
      Will\_Godfrey\_Companion/0019-Warm Square Swell \\
   \textbf{0091-Pad 3 polysynth} &
      Dual/0065-Dream of the Saw \\
   \textbf{0092-Pad 4 choir} &
      Alex\_J/0100-Choir Pad \\
   \textbf{0093-Pad 5 bowed} &
      The\_Mysterious\_Bank/0004-trance\_strings\_pad \\
   \textbf{0094-Pad 6 metallic} &
      The\_Mysterious\_Bank/0011-dreaming\_bells \\
   \textbf{0095-Pad 7 halo} &
      Alex\_J/RadioPulsePad \\
   \textbf{0096-Pad 8 sweep} &
      Pads/0011-lightbeam \\
   \textbf{0097-FX 1 rain} &
      The\_Mysterious\_Bank/0037-the\_rain \\
   \textbf{0098-FX 2 soundtrack} &
      The\_Mysterious\_Bank/0038-falling\_stars \\
   \textbf{0099-FX 3 crystal} &
      Will\_Godfrey\_Companion/0006-Tinkle Bell \\
   \textbf{0100-FX 4 atmosphere} &
      The\_Mysterious\_Bank/0038-the\_starting\_machine \\
   \textbf{0101-FX 5 brightness} &
      Noises/0014-droplets for chords \\
   \textbf{0102-FX 6 goblins} &
      Noises/0002-Ioioioioioi \\
   \textbf{0103-FX 7 echoes} &
      Noises/0072-Cave Gates \\
   \textbf{0104-FX 8 sci-fi} &
      The\_Mysterious\_Bank/0031-etrange\_sound \\
   \textbf{0105-Sitar} &
      Guitar/0097-Space Guitar \\
   \textbf{0106-Banjo} &
      Guitar/0065-Clean Guitar1 \\
   \textbf{0107-Shamisen} &
      Plucked/0034-Plucked String2 \\
   \textbf{0108-Koto} &
      Plucked/0003-Plucked 3 \\
   \textbf{0109-Kalimba} &
      Plucked/0001-Plucked 1 \\
   \textbf{0110-Bagpipe} &
      Reed\_and\_Wind/0033-Reed 1 \\
   \textbf{0111-Fiddle} &
      Laba170bank/0055-DevilsFiddle2 \\
   \textbf{0112-Shanai} &
      Reed\_and\_Wind/0035-Reed 3 \\
   \textbf{0113-Tinkle Bell} &
      Bells/0011-Glass bells \\
   \textbf{0114-Agogo} &
      The\_Mysterious\_Bank/0028-snare \\
   \textbf{0115-Steel Drums} &
      C\_Ahlstrom/Add\_Pseudo\_Steel\_Drums\_2 \\
   \textbf{0116-Woodblock} &
      The\_Mysterious\_Bank/0028-snare \\
   \textbf{0117-Taiko Drum} &
      The\_Mysterious\_Bank/0028-snare \\
   \textbf{0118-Melodic Tom} &
      The\_Mysterious\_Bank/0028-snare \\
   \textbf{0119-Synth Drum} &
      The\_Mysterious\_Bank/0028-snare \\
   \textbf{0120-Reverse Cymbal} &
      The\_Mysterious\_Bank/0028-snare \\
   \textbf{0121-Guitar Fret Noise} &
      The\_Mysterious\_Bank/0028-snare \\
   \textbf{0122-Breath Noise} &
      The\_Mysterious\_Bank/0028-snare \\
   \textbf{0123-Seashore} &
      Noises/0008-Wind and Surf \\
   \textbf{0124-Bird Tweet} &
      The\_Mysterious\_Bank/0028-snare \\
   \textbf{0125-Telephone Ring} &
      The\_Mysterious\_Bank/0028-snare \\
   \textbf{0126-Helicopter} &
      The\_Mysterious\_Bank/0028-snare \\
   \textbf{0127-Applause} &
      The\_Mysterious\_Bank/0028-snare \\
   \textbf{0128-Gunshot} &
      Noises/0018-Gun \\
   \textbf{0129-Drum Kit} &
      C\_Ahlstrom/Natural\_Drum\_Kit\_Basic \\
\end{longtable}

%-------------------------------------------------------------------------------
% vim: ts=3 sw=3 et ft=tex
%-------------------------------------------------------------------------------


   Presumably, this basic bank could be improved enough to be useful
   for most music.  Alternative (and better) banks could be created, as
   well.

   One thing to note about the instruments.  When copied from the original
   directory to the \texttt{gm-basic} directory, each instrument copied
   retains its name, of course.  To make it less confusing to use in
   \textsl{Yoshimi}, then, one should rename each instrument to its GM name.
   To do that, right click on the name, change it, and then save the
   instrument to the same file from which it was loaded.  Be careful!
   It is easy to make a mistake!  

   And we have not yet renamed any of them!

\subsection{Root Paths}
\label{subsec:cookbook_banks_root_paths}

   The first thing to do is to add the yoshimi-cookbook \texttt{banks}
   directory to your setup.

   Run \textsl{Yoshimi}, and navigate the following user-interface path:
   \textsl{Menu / Instrument / Show Root Paths ...}

   Then click \textbf{Add root directory...}.  Navigate to where the
   yoshimi-cookbook project is stored and add the \texttt{yoshimi/banks}
   directory.  The result should be something like the following:

\begin{figure}[H]
   \centering 
   \includegraphics[scale=1.0]{menu/Instrument/bank_root_paths.png}
   \caption{Bank Root Paths}
   \label{fig:cookbook_bank_root_paths}
\end{figure}

   Click on the new directory.
   \index{GM!bank}
   It has ID = 3 in that diagram.  We will refer to this value as
   the "banks path".
   Now click the \textbf{Make current} button.
   Verify that it now has the asterisk. 
   Click the \textbf{Save and Close} button.

   Now let's open the "gm-basic" bank.
   Run \textsl{Yoshimi}, and navigate the following user-interface path:
   \textsl{Menu / Instrument / Show Banks ...}

   In the matrix of banks, you should see "gm-basic" somewhere.
   (We also have a "demo" bank in place.)

\begin{figure}[H]
   \centering 
   \includegraphics[scale=0.75]{menu/Instrument/demo_gm-basic_banks.png}
   \caption{Two Banks, Demo and Basic GM}
   \label{fig:cookbook_bank_demo_gm_basic}
\end{figure}
   
   \index{directories!GM basic bank}
   Click on the "gm-basic" bank.  The larger dialog below will be shown.
   \index{current bank}
   Note that this action also makes "gm-basic" the current bank.
   This setting will be preserved across a restart of \textsl{Yoshimi}.

\begin{figure}[H]
   \centering 
   \includegraphics[scale=0.85]{menu/Instrument/banks_gm_basic.png}
   \caption{A General MIDI Basic Bank}
   \label{fig:cookbook_bank_basic_bank}
\end{figure}

   Remember that these banks have GM names; the original files used to
   create each one are listed in the spreadsheet mentioned in the previous
   section.

   \index{drum kit}
   \index{GM!drum kit}
   Also note the drum kit, with an ID of 129.  Normally, drum kits might be
   stored in a bank of drum kits, but here we have a GM-compliant drum kit,
   compliant in the sense that most of the keys are mapped correctly, and
   all keys will play \textsl{something}.

\subsection{Using a Basic GM Bank}
\label{subsec:cookbook_banks_using_basic_bank}

   \index{current root}
   Once the current root path has been set (e.g. to
   \texttt{~/yoshimi-cookbook/yoshimi/banks}) and
   \index{current bank}
   the current bank has been set (e.g. to \texttt{gm-basic}),
   then a MIDI file can select instruments 1 to 128 from the
   current bank using a Program Change event followed by the desired
   instrument's number re 0.

   The bank and root paths can be saved in the \textsl{Yoshimi} state.
   But the MIDI file can also start with bank-selection events, and change
   the current bank using Bank Select LSB (CC32) and Bank Select MSB (CC0)
   events.

%-------------------------------------------------------------------------------
% vim: ts=3 sw=3 et ft=tex
%-------------------------------------------------------------------------------


% %-------------------------------------------------------------------------------
% cookbook_concepts
%-------------------------------------------------------------------------------
%
% \file        cookbook_concepts.tex
% \library     Documents
% \author      Chris Ahlstrom
% \date        2015-07-15
% \update      2015-07-15
% \version     $Revision$
% \license     $XPC_GPL_LICENSE$
%
%     Provides the concepts.
%
%-------------------------------------------------------------------------------

\section{Concepts}
\label{sec:concepts}

   This section, like its counterpart in our \textsl{Yoshimi User Manual},
   presents some useful concepts, while keeping them out of the way.

\subsection{Concepts / Terms}
\label{subsec:concepts_terms}

   This section doesn't provide comprehensive coverage of terms.  It
   covers mainly terms that puzzled the author at first.

\subsubsection{Concepts / Terms / cent}
\label{subsubsec:concepts_terms_cent}

   The \textbf{cent}
   \index{cent}
   is a logarithmic unit of measure used for musical
   intervals.  Twelve-tone equal temperament divides the octave into 12
   semitones of 100 cents each. Typically, cents are used to measure
   extremely small finite intervals, or to compare the sizes of comparable
   intervals in different tuning systems.
   The interval of one cent is much too small to be heard between
   successive notes.

   Since the detuning provided in \textsl{Yoshimi} is based primarily on
   cents (and octaves), it pays to understand cents.  If a given frequency
   \texttt{f'} is offset from another frequency \texttt{f}, the
   relationships between them in semitones are:

   \[f' = f * 2^s/12\]

   \[s = 12 log (f'/f) / log 2\]

   In cents, these relationships become:

   \[f' = f * 2^s/1200\]

   \[s = 1200 log (f'/f) / log 2\]

   These relationships hold whether \texttt{f'} is less than or greater than
   \texttt{f}.  They provide an easy way to determine how much to detune a
   frequency in \textsl{Yoshimi}.

%-------------------------------------------------------------------------------
% vim: ts=3 sw=3 et ft=tex
%-------------------------------------------------------------------------------


\section{Summary}
\label{sec:summary}

   In summary, we can say that you will absolutely love 
   cooking with \textsl{Yoshimi}.

% References

%-------------------------------------------------------------------------------
% cookbook_references
%-------------------------------------------------------------------------------
%
% \file        cookbook_references.tex
% \library     Documents
% \author      Chris Ahlstrom
% \date        2015-07-12
% \update      2016-03-13
% \version     $Revision$
% \license     $XPC_GPL_LICENSE$
%
%     Provides the References section of yoshimi-cookbook.tex.  Rather
%     than use the bibtex package, our small set of references uses a
%     simpler method.
%
%-------------------------------------------------------------------------------

\section{References}
\label{sec:cookbook_references}

   The \textsl{Yoshimi} cookbook reference list.

\begin{thebibliography}{99}

   \bibitem{bankrootupgrades}
   Will J. Godfrey
   \emph{A discussion of making Bank/Root specifications more regular.}
   \url{http://sourceforge.net/p/yoshimi/mailman/message/33200765/}
   2014.

   \bibitem{bellsimple}
   Blair School of Music, Vanderbilt University.
   \emph{Creating a Simple Bell}
   \url{http://computermusicresource.com/Simple.bell.tutorial.html}
   2012?

   \bibitem{bellspectrum}
   FM 8 Tutorials
   \emph{Spectrum of a Simple Bell}
   \url{http://www.fm8tutorials.com/wp-content/uploads/2012/03/Bell-spectrum.png}
   2012?

   \bibitem{ringmodulator}
   LinuxMusicians newsgroup
   \emph{Ring Modulation in ZynAddSubFX}
   \url{http://linuxmusicians.com/viewtopic.php?f=1&t=8178}
   2012.

   \bibitem{sequencer64}
   Chris Ahlstrom.
   \emph{Sequencer64}
   \url{https://github.com/ahlstromcj/sequencer64/}
   2016.

   \bibitem{sequencer64doc}
   Chris Ahlstrom.
   \emph{The Sequencer64 User Manual.}
   \url{https://github.com/ahlstromcj/sequencer64-doc/}
   2016.

   \bibitem{scala}
   Manuel Op de Coul <\url{coul@huygens-fokker.org}>
   \emph{The Scala Musical Tuning Application.}
   \url{http://www.huygens-fokker.org/scala/}
   Scala is a powerful software tool for experimentation with musical
   tunings, such as just intonation scales, equal and historical
   temperaments, microtonal and macrotonal scales, and non-Western scales.
   2014.

   \bibitem{sharphall}
   Sharphall
   \emph{How to create drum sounds in ZynAddSubFX or Yoshimi, Part 1}
   \url{http://sharphall.org/docs/zynaddsubfx\_yoshimi\_drum\_tutorial.php}
   Never got continued, unfortunately.

   \bibitem{synthsecrets}
   Gordon Reid
   \emph{Synth Secrets:  Creative Synthesis}
   \url{http://www.soundonsound.com/sos/allsynthsecrets.htm}
   1999-2004.

   \bibitem{yoshimi}
   Yoshimi team \url{abrolag@users.sourceforge.net}
   \emph{The download site for the Yoshimi software synthesizer.}
   \url{http://yoshimi.sourceforge.net/}
   2016.

   \bibitem{yoshimi2}
   Yoshimi team
   \emph{The alternate location for the Yoshimi source-code.}
   \url{https://github.com/abrolag/yoshimi/}
   2016.

   \bibitem{yoshimidoc}
   Chris Ahlstrom
   \emph{A Yoshimi User Manual.}
   \url{https://github.com/ahlstromcj/yoshimi-doc/}
   2016.

   \bibitem{yoshimicookbook}
   Chris Ahlstrom
   \emph{A Yoshimi Cookbook.}
   \url{https://github.com/ahlstromcj/yoshimi-cookbook/}
   2016.
   
   \bibitem{yoshimidrums}
   Barney Holmes (djbarney)
   \emph{Generating synthesised drums and percussion in Linux using Yoshimi
   or ZynAddSubFX.}
   \url{https://djbarney.wordpress.com/2013/10/27/generating-synthesised-drums-and-percussion-in-linux-using-yoshimi/}
   2013.

\end{thebibliography}

%-------------------------------------------------------------------------------
% vim: ts=3 sw=3 et ft=tex
%-------------------------------------------------------------------------------


\printindex

\end{document}

%-------------------------------------------------------------------------------
% vim: ts=3 sw=3 et ft=tex
%-------------------------------------------------------------------------------
