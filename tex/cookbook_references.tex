%-------------------------------------------------------------------------------
% cookbook_references
%-------------------------------------------------------------------------------
%
% \file        cookbook_references.tex
% \library     Documents
% \author      Chris Ahlstrom
% \date        2015-07-12
% \update      2016-03-13
% \version     $Revision$
% \license     $XPC_GPL_LICENSE$
%
%     Provides the References section of yoshimi-cookbook.tex.  Rather
%     than use the bibtex package, our small set of references uses a
%     simpler method.
%
%-------------------------------------------------------------------------------

\section{References}
\label{sec:cookbook_references}

   The \textsl{Yoshimi} cookbook reference list.

\begin{thebibliography}{99}

   \bibitem{bankrootupgrades}
   Will J. Godfrey
   \emph{A discussion of making Bank/Root specifications more regular.}
   \url{http://sourceforge.net/p/yoshimi/mailman/message/33200765/}
   2014.

   \bibitem{bellsimple}
   Blair School of Music, Vanderbilt University.
   \emph{Creating a Simple Bell}
   \url{http://computermusicresource.com/Simple.bell.tutorial.html}
   2012?

   \bibitem{bellspectrum}
   FM 8 Tutorials
   \emph{Spectrum of a Simple Bell}
   \url{http://www.fm8tutorials.com/wp-content/uploads/2012/03/Bell-spectrum.png}
   2012?

   \bibitem{ringmodulator}
   LinuxMusicians newsgroup
   \emph{Ring Modulation in ZynAddSubFX}
   \url{http://linuxmusicians.com/viewtopic.php?f=1&t=8178}
   2012.

   \bibitem{sequencer64}
   Chris Ahlstrom.
   \emph{Sequencer64}
   \url{https://github.com/ahlstromcj/sequencer64/}
   2016.

   \bibitem{sequencer64doc}
   Chris Ahlstrom.
   \emph{The Sequencer64 User Manual.}
   \url{https://github.com/ahlstromcj/sequencer64-doc/}
   2016.

   \bibitem{scala}
   Manuel Op de Coul <\url{coul@huygens-fokker.org}>
   \emph{The Scala Musical Tuning Application.}
   \url{http://www.huygens-fokker.org/scala/}
   Scala is a powerful software tool for experimentation with musical
   tunings, such as just intonation scales, equal and historical
   temperaments, microtonal and macrotonal scales, and non-Western scales.
   2014.

   \bibitem{sharphall}
   Sharphall
   \emph{How to create drum sounds in ZynAddSubFX or Yoshimi, Part 1}
   \url{http://sharphall.org/docs/zynaddsubfx\_yoshimi\_drum\_tutorial.php}
   Never got continued, unfortunately.

   \bibitem{synthsecrets}
   Gordon Reid
   \emph{Synth Secrets:  Creative Synthesis}
   \url{http://www.soundonsound.com/sos/allsynthsecrets.htm}
   1999-2004.

   \bibitem{yoshimi}
   Yoshimi team \url{abrolag@users.sourceforge.net}
   \emph{The download site for the Yoshimi software synthesizer.}
   \url{http://yoshimi.sourceforge.net/}
   2016.

   \bibitem{yoshimi2}
   Yoshimi team
   \emph{The alternate location for the Yoshimi source-code.}
   \url{https://github.com/abrolag/yoshimi/}
   2016.

   \bibitem{yoshimidoc}
   Chris Ahlstrom
   \emph{A Yoshimi User Manual.}
   \url{https://github.com/ahlstromcj/yoshimi-doc/}
   2016.

   \bibitem{yoshimicookbook}
   Chris Ahlstrom
   \emph{A Yoshimi Cookbook.}
   \url{https://github.com/ahlstromcj/yoshimi-cookbook/}
   2016.
   
   \bibitem{yoshimidrums}
   Barney Holmes (djbarney)
   \emph{Generating synthesised drums and percussion in Linux using Yoshimi
   or ZynAddSubFX.}
   \url{https://djbarney.wordpress.com/2013/10/27/generating-synthesised-drums-and-percussion-in-linux-using-yoshimi/}
   2013.

\end{thebibliography}

%-------------------------------------------------------------------------------
% vim: ts=3 sw=3 et ft=tex
%-------------------------------------------------------------------------------
