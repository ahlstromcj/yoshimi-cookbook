%-------------------------------------------------------------------------------
% cookbook_instruments
%-------------------------------------------------------------------------------
%
% \file        cookbook_instruments.tex
% \library     Documents
% \author      Chris Ahlstrom
% \date        2015-07-12
% \update      2015-07-16
% \version     $Revision$
% \license     $XPC_GPL_LICENSE$
%
%     Provides the cookbooks_instruments section of yoshimi-cookbook.tex.
%
%-------------------------------------------------------------------------------

\section{Creating Instruments}
\label{sec:cookbook_instruments}

   One of our goals in using \textsl{Yoshimi} is to support
   \textsl{General MIDI} (\textsl{GM})
   to the greatest extent possible.

   However, no banks have been created with GM in mind.  And many of the
   instruments, though given names that indicate what they are intended to
   be, fall well short of being recognizable per their name; they should be
   doable with a complex synthesizer like \textsl{Yoshimi}.
  
   It is true that there are a vast number of patches out there.  The author
   has attempted a survey of them, and the task is all but impossible.
   Still, many candidates have been identified.  Other candidates might be
   suitable with a little tweaking.

   Here are a number of categories of instruments for which we want to
   assemble an improved set of instruments.

   \begin{enumber}
      \item \textbf{Bells}
      \item \textbf{Ethnic}
      \item \textbf{Drums}
      \item \textbf{Effects}
      \item \textbf{Piano}
      \item \textbf{Leads}
      \item \textbf{Guitar}
      \item \textbf{Strings} (individual and ensemble)
      \item \textbf{Bass}
      \item \textbf{Saxophones}
   \end{enumber}

   For these recipes, the \texttt{banks} directories will
   be stored in the following directory of this project:
   \index{directories!demo bank}

   \begin{verbatim}
      yoshimi/banks
      yoshimi/banks/demo
   \end{verbatim}

\subsection{Bells}
\label{subsec:cookbook_instruments_bells}

   The bells patches we've heard so far are nice, but a bit anemic.
   
   \index{ring modulation}
   Good bell patches are easier with ring modulation, done right.  We're not
   sure if there are any such patches extant; please send us to them if
   there are some.

   \index{bells}
   In the meantime, creating bells is a good excuse to master
   \textsl{Yoshimi}'s ring modulator.
   However, we will first learn how to create a reasonable, clangy bell
   using just a few voices in an ADDsynth part, and no need for modulation.

\subsubsection{Bells by Voices}
\label{subsec:cookbook_instruments_bells_by_voices}

   The following table comes from a tutorial (\cite{bellsimple}).  Along
   with a spectrum shown in reference \cite{bellspectrum}, it allows us to
   recreate a simple, but realistic bell.
   \index{bells!spectrum}
   In this table, \texttt{F} represents the fundamental frequency, i.e. the
   note being played.

\label{table:simple_bell_tones}
\begin{longtable}{c c c c}
   \caption{Simple Bell Tones} \\
   \hline
      \textbf{Wave Number} &
      \textbf{Frequency} &
      \textbf{Cents Offset} &
      \textbf{Relative Amplitude} \\
   \hline
   \endfirsthead

   1 &  0.56F &  -1000 &   0.5   \\
   2 &  0.92F &  -140  &   1.0   \\
   3 &  1.19F &  +300  &   0.5   \\
   4 &  1.71F &  +930  &   0.25  \\
   5 &  2.00F &  +1200 &   0.125 \\
   6 &  2.74F &  +1745 &   0.125 \\
   7 &  3.00F &  +1901 &   0.125 \\
   8 &  3.76F &  +2290 &   0.125 \\
   9 &  4.00F &  +2400 &         \\
\end{longtable}

   Note that the frequencies are relative to the fundamental frequency (F).
   Also note that wave 2 (close to F) can be missing, and the sound still is
   bell-like.

   \index{files!bells addsynth}
   The file \texttt{yoshimi/banks/demo/Bells-simple-addsynth.xiz} is the
   result of following the steps below.  We start, as usual, with a
   newly-started \textsl{Yoshimi} instance.

   \begin{enumber}
      \item Open the ADDsynth editing window by clicking the \textbf{Edit}
         button in the bottom panel, and then clicking the ADDsynth
         \textbf{Edit} button in the edit window.
      \item Click on the \textbf{Show Voice Parameters} button.
         Note that it is \textbf{Current Voice 1}, and it should be enabled.
      \item For voice 1, make the following settings:
         \begin{enumber}
            \item \textbf{Octave}: Set to 0.
            \item \textbf{Detune Type}: Set to E1200cents.
            \item \textbf{FREQUENCY Detune}: Set to -1000 approximately.
         \end{enumber}
      \item Go to voice 2, and make the following settings:
         \begin{enumber}
            \item \textbf{Octave}: Set to 0.
            \item \textbf{Detune Type}: Set to E1200cents.
            \item \textbf{FREQUENCY Detune}: Set to -140 approximately.
         \end{enumber}
      \item Go to voice 3, and make the following settings:
         \begin{enumber}
            \item \textbf{Octave}: Set to 0.
            \item \textbf{Detune Type}: Set to E1200cents.
            \item \textbf{FREQUENCY Detune}: Set to 300 approximately.
         \end{enumber}
      \item Go to voice 4, and make the following settings:
         \begin{enumber}
            \item \textbf{Octave}: Set to 0.
            \item \textbf{Detune Type}: Set to E1200cents.
            \item \textbf{FREQUENCY Detune}: Set to 930 approximately.
         \end{enumber}
      \item Go to voice 5, and make the following settings:
         \begin{enumber}
            \item \textbf{Octave}: Set to 1.
            \item \textbf{Detune Type}: Set to Default.
            \item \textbf{FREQUENCY Detune}: Set to 0.
         \end{enumber}
      \item Go to voice 6, and make the following settings:
         \begin{enumber}
            \item \textbf{Octave}: Set to 1.
            \item \textbf{Detune Type}: Set to E1200cents.
            \item \textbf{FREQUENCY Detune}: Set to 545 approximately.
         \end{enumber}
      \item Go to voice 7, and make the following settings:
         \begin{enumber}
            \item \textbf{Octave}: Set to 1.
            \item \textbf{Detune Type}: Set to E1200cents.
            \item \textbf{FREQUENCY Detune}: Set to 700 approximately.
         \end{enumber}
      \item Go to voice 8, and make the following settings:
         \begin{enumber}
            \item \textbf{Octave}: Set to 1.
            \item \textbf{Detune Type}: Set to E1200cents.
            \item \textbf{FREQUENCY Detune}: Set to 1090 approximately.
         \end{enumber}
   \end{enumber}

   These settings then end up roughly matching the settings of the first 8
   waves in \tableref{table:simple_bell_tones}.
   \index{bells}
   This instrument isn't perfect.  It's not quite equally tempered, though
   close.  The character of the tone changes a bit as the notes get higher.
   One can fiddle with the relative amplitudes of the various voices to
   change the character of this sound.

\subsubsection{Ring Modulation with 440 Hz Tone}
\label{subsec:cookbook_instruments_ring_mod_440}

   Now for an initial demonstration of ring modulation.
   This demonstration does not quite create a bell tone, but does show
   the sound of modulation.

   Start with a fresh \textsl{Yoshimi} and a cleared instrument ("Simple
   Sound").  Open the virtual keyboard using the \textbf{virKbd} button.
   Click a key and verify that you can hear a tone.
   \index{"C" note}
   \index{comma key}
   We'll use the middle C
   key (the "comma" on the PC keyboard) as a reference.  We will call it
   the "C" note.

   The following steps will set up two tones, voice 1 and voice 2, and voice
   2 will use voice 1 as an external modulator.
   Note that you can accomplish most of these steps by loading the project
   file
   \index{bells}
   \index{files!bells 440}
   \texttt{yoshimi/banks/demo/Bells-440-ring-modulation.xiz}, but use
   that only as a last resort.

   \begin{enumber}
      \item Open the ADDsynth editing window by clicking the \textbf{Edit} button
         in the bottom panel, and then clicking the ADDsynth \textbf{Edit}
         button in the edit window.
      \item In the \textbf{Amplitude Env} sub-panel, increase the
         \textbf{D.dt} and \textbf{R.dt} to give the current
         sound a nice slow decay.
      \item Click on the \textbf{Show Voice Parameters} button.
         Note that it is \textbf{Current Voice 1}, and it should be enabled.
      \item Switch to \textbf{Current Voice} number 2 and enable it.
         Play the "C" note, and observe that it is the same frequency, but
         louder.
      \item Move the \textbf{FREQUENCY Detune} slider a bit, and play the "C"
         note.  It should sound the same as before, but change slowly in
         amplitude, as heard and as seen on the \textbf{VU meter}.
         Try to set the detune back to 0; this is easier if you highlight
         the tuning knob and use the left or right arrow keys.
      \item In the \textbf{MODULATOR} section of voice 2, for \textbf{Type},
         select the \textbf{RING} value.  (However, feel free to select one
         of the other modulators, to experiment, once you've mastered
         the ring modulator.)  Press the "C" key again, and notice
         that the tone character changes a bit.  This is due to the internal
         modulator.
      \item For \textbf{External Mod.} for voice 2, select
         \textbf{Ext.M} 1, to use the voice 1 as the internal modulator.
         The "C" note may change in character, but only slightly.
         Apparently the default internal modulator is the same as the
         default external voice 1 waveform.
      \item To actually hear some modulation, we have to separate the
         frequencies of voice 1 and voice 2.  Click the \textbf{440Hz}
         check-box in the \textbf{FREQUENCY} section of voice 1.  Press the
         "C" key and verify hearing a two-tone signal, somewhat like a phone
         tone.
      \item Now go back to voice 2 and click the \textbf{Change} button to
         bring up the ADDsynth oscillator dialog.
      \item Move the slider to maximum for harmonic 10.  Press the "C" key
         and verify the new sound (a bit like a car horn).
         Set the sliders back to 0, and "C" will be a single tone again.
      \item Change the \textbf{Octave} values of voice 2 in its
         \textbf{FREQUENCY} section and listen to the effects.
   \end{enumber}

   Now we need to see if we can apply modulation across instruments.
   Sadly, this does not seem to be possible.

   Increase the \textbf{D.dt} and \textbf{R.dt} values of the main
   \textbf{Amplitude Env} to give this sound the onset and decay of a bell,
   and it then sounds less abstract, and more like a bell.
   Of course, this kind of bell is even less tunable than the simple
   bell of the previous section.

   Another thing to try with this setup is to simply change voice 2 to use
   different types of modulators besides \textbf{RING}.
   \textbf{MORPH} sounds basically identical to \textbf{RING}.
   \textbf{PM} seems to expose higher harmonics, making the sound louder and
   brighter.
   \textbf{FM} sounds similar to PM, but softer and smoother.
   \textbf{PITCH} is disabled.

   Another experiment is to disable the modulator (voice 1 here) and see how
   that changes the sound; all it should do is drop voice 1 from the
   spectrum -- voice 1 will still be used as the modulator.

   Finally, by adding a slow decay to this sound, it becomes amazingly more
   bell-like.

\subsubsection{Complex Bells by Ring Modulation}
\label{subsec:cookbook_instruments_bells_by_ring_mod}

   The next step is to make the bells more complex, by combining the methods
   of the previous two sections.
   Recall \tableref{table:simple_bell_tones}.
   It shows the 9 frequencies in the simple bell spectrum, though we could
   define only 8 of them.  How can we best add extra frequencies?
   \index{bells}
   We can ring-modulate the higher frequencies against one of the lower
   frequencies.

\label{table:ring_mod_bell_tones}
\begin{longtable}{l l l l l}
   \caption{Ring Modulation Bell Tones} \\
   \hline
      \textbf{Wave Number} &
      \textbf{Frequency} &
      \textbf{Mod Frequency} &
      \textbf{f2-f1} &
      \textbf{f2+f1} \\
   \hline
   \endfirsthead

   1 &  0.56F &  ----- &  -----   &  -----   \\
   2 &  0.92F &  0.56F &  0.36F   &  1.48F   \\
   3 &  1.19F &  0.56F &  0.63F*  &  1.75F*  \\
   4 &  1.71F &  0.56F &  1.15F*  &  2.27F   \\
   5 &  2.00F &  0.56F &  1.44F   &  2.56F   \\
   6 &  2.74F &  0.56F &  2.18F   &  3.30F   \\
   7 &  3.00F &  0.56F &  2.44F   &  3.56F   \\
   8 &  3.76F &  0.56F &  3.20F   &  4.32F   \\
\end{longtable}

   The asterisk marks frequencies that are close to existing
   frequencies.   Luckily, there are only three of them, so our
   modulation should add a good number of frequencies.

   \begin{enumber}
      \item Load the file \texttt{yoshimi/banks/demo/Bells-simple-addsynth.xiz}
         to save a lot of steps.  The next steps add voice 1 as a ring
         modulator for voices 2 through 8.
      \item Open the ADDsynth editing window by clicking the \textbf{Edit}
         button in the bottom panel, and then clicking the ADDsynth
         \textbf{Edit} button in the edit window.
      \item Click on the \textbf{Show Voice Parameters} button.
         Note that it is \textbf{Current Voice 1}, and it should be enabled.
      \item Go to voice 2 and do the following steps:
         \begin{enumber}
            \item In the \textbf{MODULATOR} section (greyed out), change the
               \textbf{Type} from \textbf{OFF} to \textbf{RING}.
            \item Changes the \textbf{External Mod.} dropdown from
               \textbf{Off} to \textbf{ExtMod. 1}.
         \end{enumber}
      \item Go to voice 3 and repeat those steps.  Note how all the voices
         below the current voice become available as modulators.
   \end{enumber}

   We saved the result in the file
   \index{files!bells ring mod}
   \texttt{yoshimi/banks/demo/Bells-ringmod-addsynth.xiz} for safe-keeping.

   QUESTION:  If one loads and instrument and tinkers with it, but do not
   save it, does \textsl{Yoshimi} save it on exit anyway?

\subsection{Ethnic}
\label{subsec:cookbook_instruments_ethnic}

   We've found a simple steel drum instruments, but think we might do better,
   creating one using ADDsynth and one using PADsynth.

   Instruments we have not found, and would like to synthesize, are:
   bagpipes and arabic pipes.

\subsubsection{Ethnic / Steel Drums}
\label{subsubsec:cookbook_instruments_ethnic_steeldrums}

   \index{steel drums}
   There is a steel-drum instrument that ships with \textsl{Yoshimi}:
   \texttt{/usr/share/yoshimi/banks/The\_Mysterious\_Bank/0122-pseudo\_steeldrums.xiz}.
   It is an ADDsynth module comprised of three voices:

   \begin{enumber}
      \item A \textbf{Unison}-enabled voice of \textbf{Size} = 10 and a
         \textbf{Frequency Spread} of 44.6 cents.
      \item Another voice that is exactly the same as the first, except that
         it has its \textbf{Amplitude Env} sub-panel enabled, to add more
         volume and character to the instrument, it is stronger on the
         right, and, most importantly, an octave higher.
      \item Another voice that is exactly the same as the second, 
         except it is an octave lower than voice 0.
   \end{enumber}

   If voice 2 and 3 are disabled, the instrument still sounds reminiscent of
   steel drums, so obviously the overall amplitude envelope is important.

   Can we do better?  Well, the instrument above sounds too pristine.
   We should be able to add some "dirt" to the instrument to make it sound
   more lively.

\begin{figure}[H]
   \centering 
   \includegraphics[scale=0.85]{contrib/steel_drum_spectrum.png}
   \caption{Typical Steel Drum Spectrum}
   \label{fig:cookbook_bank_steeldrum_spectrum}
\end{figure}

   Taking a cue from this figure, our steel drums extend the original
   by adding a couple more tones at octave intervals.  Also, some slight
   detuning was introduced to add to the flavor.
   We could probably add a couple more, and carefully contour their
   amplitude levels to match the spectrum levels shown above.

   To hear the ADDsynth steel drum sound, load the file
   \texttt{yoshimi/banks/demo/Add\_Pseudo\_Steel\_Drums.xiz}.

   Not content with that, with our hands behind our back, we pull
   SUBsynth from a hat.  The SUBsynth settings for a steel drum are shown in
   the following figure:

\begin{figure}[H]
   \centering 
   \includegraphics[scale=0.75]{demo/steeldrums-subsynth-editor.png}
   \caption{Steel Drum SUBsynth Configuration}
   \label{fig:cookbook_bank_steeldrum_subsynth}
\end{figure}

   Note the top box of slider controls.  It sets the amplitudes of
   the harmonics, and should vaguely resemble the spectrum diagram.
   The lower box of slider controls sets the bandwidth of each harmonic,
   with the fundamental frequency being very narrow.

   To hear the SUBsynth steel drum sound, load the file
   \texttt{yoshimi/banks/demo/Sub\_Pseudo\_Steel\_Drums.xiz}.

\subsection{Drums}
\label{subsec:cookbook_instruments_drums}

   We want a decent drum kit that attempts to fill in the gaps for a
   GM-compliant drum kit with solid sounds, with
   the help of an existing kit.

   It turns out that a "Natural Drum Kit", which we'd found separately on
   the Internet a long time ago, is now part of the instruments
   installed with \textsl{Yoshimi}.
   \index{drum kit}
   \index{GM!drum kit}
   But long ago we used some of the sounds
   from various kits to create our own "natural drum kit", and extended some
   of the sounds across more (pitched) keys so that any MIDI drum note would
   produce \textsl{some} sound.  We also made sure the sounds were laid out
   in GM format as much as possible.

   Fire up \textsl{Yoshimi} and load the instrument stored in
   \texttt{yoshimi/banks/demo/Natural\_Drum\_Kit\_Basic.xiz}, and we'll walk
   through it.

\begin{figure}[H]
   \centering 
   \includegraphics[scale=0.85]{demo/natural_drum_kit_from_ds_2.png}
   \caption{Natural Drum Kit from DS 2}
   \label{fig:cookbook_bank_natural_drum_kit}
\end{figure}

   \textbf{Item 1} is the master control for the whole kit, determining the
   range of keys that it covers and the effect (if any) it goes through.

   \textbf{Item 2} and \textbf{Item 3} provide the two parts of the
   "natural" snare drum.  Both parts are composed of an ADDsynth and a
   SUBsynth section, and they provide 3 pitches of snare.

   \index{natural drum!snare}
   The "Snare - Stick + Snares" sections....

   \index{natural drum!hihats}
   \index{natural drum!cymbal}
   \index{natural drum!side stick}
   \index{natural drum!toms}
   \index{natural drum!extra toms}
   \index{natural drum!bass drums}
   \index{natural drum!xxxxxxx}
   \index{natural drum!xxxxxxx}
   \index{natural drum!xxxxxxx}
   MORE DESCRIPTIONS
   MORE DESCRIPTIONS
   MORE DESCRIPTIONS
   MORE DESCRIPTIONS
   MORE DESCRIPTIONS

   Now, without defining more than one drum kit, we have only about 15
   "drums" available to us in \textsl{Yoshimi}.  So we filled in the missing
   drums with more "toms", just so some sound will be made.  The frequencies
   of the upper toms get pretty crazy!  Here's a diagram of the keyboard
   layout.  Correlate it with
   \figureref{fig:cookbook_bank_natural_drum_kit}, to understand the
   abbreviations.

\begin{figure}[H]
   \centering 
   \includegraphics[scale=0.75]{demo/natural_drum_keyboard_layout.png}
   \caption{Natural Drum Kit Keyboard Layout}
   \label{fig:cookbook_bank_natural_drum_kit_layout}
\end{figure}

   For your reference, here is the full GM drum layout.  The diagram is
   taken from WikiMedia.org.

\begin{figure}[H]
   \centering 
   \includegraphics[scale=0.75]{demo/GMStandardDrumMap.png}
   \caption{General MIDI Drum Kit Keyboard Layout}
   \label{fig:cookbook_bank_general_drum_kit_layout}
\end{figure}

\subsection{Effects}
\label{subsec:cookbook_instruments_effects}

   This section documents the various "effects" instruments we've created.

\subsubsection{Effects / Dial Tones}
\label{subsubsec:cookbook_instruments_effects_dialtones}

   We've created a nice dial-tone effect that we'll describe here.
   Dial tones consist of two notes, as shown by the
   \textbf{Low F} and \textbf{High F} columns in the following table.

\label{table:effect_dial_tones}
\begin{longtable}{l l l l l l l l l l}
   \caption{DTMF Frequencies Table} \\
   \hline
      \textbf{Tag} &
      \textbf{DTMF} &
      \textbf{Kit\#} &
      \textbf{MIDI\#} &
      \textbf{Low} &
      \textbf{Low F} &
      \textbf{Actual F} &
      \textbf{High} &
      \textbf{High F} &
      \textbf{Actual F} \\
   \hline
   \endfirsthead

   1 &  1        & 5  & 53 & F3 &  697 &  705 &  F5     & 1209 &  1245  \\
   2 &  2        & 6  & 77 & F5 &  697 &  705 &  F5     & 1336 &  1337  \\
   3 &  3        & 7  & 89 & F6 &  697 &  698 &  F5     & 1477 &  1468  \\
   4 &  4        & 8  & 55 & G3 &  770 &  770 &  G5 -   & 1209 &  1236  \\
   5 &  5        & 9  & 79 & G5 &  770 &  776 &  G5 -   & 1336 &  1334  \\
   6 &  6        & 10 & 91 & G6 &  770 &  773 &  G5 -   & 1477 &  1462  \\
   7 &  7        & 11 & 57 & A3 &  852 &  855 &  G\#5 + & 1209 &  1245  \\
   8 &  8        & 12 & 81 & A5 &  852 &  868 &  G\#5 + & 1336 &  1327  \\
   9 &  9        & 13 & 93 & A6 &  852 &  866 &  G\#5 + & 1477 &  1480  \\
   * &  *        & 14 & 59 & B3 &  941 &  948 &  A\#5 + & 1209 &  1257  \\
   0 &  0        & 15 & 83 & B5 &  941 &  968 &  A\#5 + & 1336 &  1281  \\
  \# &  \#       & 16 & 95 & B6 &  941 &  950 &  A\#5 + & 1477 &  1480  \\
   A &  A        & -- & -- & A2 &  697 &  --- &  F5     & 1633 &  ---   \\
   B &  B        & -- & -- & B2 &  770 &  --- &  G5 -   & 1633 &  ---   \\
   C &  C        & -- & -- & C2 &  852 &  --- &  G\#5 + & 1633 &  ---   \\
   D &  D        & -- & -- & D2 &  941 &  --- &  A\#5 + & 1633 &  ---   \\
   b &  busy     & 2  & 71 & B4 &  480 &  472 &  B4 -   & 620  &  622   \\
   r &  ringback & 3  & 69 & A4 &  440 &  440 &  A4     & 480  &  480   \\
   d &  dialtone & 4  & 65 & F4 &  350 &  350 &  F4     & 440  &  440   \\
\end{longtable}

   This table is implemented in a \textsl{Yoshimi kit}.  Each note in the
   kit is created by making an ADDsynth instrument with two voices.  The
   lower voice generally corresponds to the note being play, with an offset,
   if needed, to achieve close to the proper frequency for the lower note of
   the DTMF tone.
   The second voice corresponds to the other note, and it is detune
   appropriately to achieve close to the proper frequency for the upper note
   of the DTMF tone.

   The following figure shows the kit dialog:

\begin{figure}[H]
   \centering 
   \includegraphics[scale=1.0]{demo/DTMF_kit_edit.png}
   \caption{Kit Edit Dialog for DTMF Kit}
   \label{fig:cookbook_instruments_dtmf_kit_edit}
\end{figure}

   To edit the kit, follow the steps below.  If desired, open the
   instrument file
   \index{files!DTMF kit}
   \texttt{yoshimi/banks/demo/DTMF\_Kit.xiz} to save some
   work.

   \begin{enumber}
      \item Open the kit editing window by clicking the \textbf{Edit}
         button in the bottom panel, and then clicking the
         \textbf{Kit Edit} button in the edit window.
      \item Make sure that the \textbf{Mode} is set to \textbf{Single}.
      \item Make sure that the \textbf{Drum mode} is enabled.
      \item For all 16 kit items, make sure that the \textbf{FX.r}
         selections are set to \textbf{OFF}.
      \item For kit items 2 to 16, enable the the \textbf{ADsynth}
         check-box.
      \item For kit items 2 to 16, perform the following procedure to set up
         the two frequencies correctly as per the table above:
         \begin{enumber}
            \item In the kit editor, click the \textbf{Name} field and enter
               the name of the DTMF tone item being edited.
            \item In the kit editor, set \textbf{Min.k} and \textbf{Max.k}
               to the value of the note that is less than or equal to the
               lower note of the item listed in the table.
            \item Click the ADDsynth \textbf{edit} button.
            \item Click on the \textbf{Show Voice Parameters} button.
               Note that it is \textbf{Current Voice 1}, and it should be
               enabled.
            \item Given the frequency for the note being edited, detune
               voice 1 to achieve the desired lower frequency.
            \item Change to voice 2, and enable it.
            \item Given the frequency for the note being edited, detune
               voice 1 to achieve the desired higher frequency.
         \end{enumber}
   \end{enumber}

   The "Actual F" values were verifed using 24-Hz resolution (at 1200 Hz)
   in the spectrum analyzer built into
   \index{Audacity}
   Audacity.
   Sometimes it took a few
   tries to get the best possible frequency.  We could list the detuning
   values in a table; for now, you can see the values we ended up using.

   The following figure shows the layout on the \textsl{Yoshimi} virtual
   keyboard:

\begin{figure}[H]
   \centering 
   \includegraphics[scale=1.0]{demo/DTMF-layout.png}
   \caption{DTMF Layout on the Keyboard}
   \label{fig:cookbook_instruments_DTMF_layout}
\end{figure}

\subsection{Piano}
\label{subsec:cookbook_instruments_piano}

   TODO.

\subsection{Leads}
\label{subsec:cookbook_instruments_leads}

   TODO.

\subsection{Guitar}
\label{subsec:cookbook_instruments_guitar}

   TODO.

\subsection{Strings} (individual and ensemble)
\label{subsec:cookbook_instruments_strings}

   TODO.

\subsection{Bass}
\label{subsec:cookbook_instruments_bass}

   TODO.

\subsection{Saxophones}
\label{subsec:cookbook_instruments_saxophones}

   TODO.

%-------------------------------------------------------------------------------
% vim: ts=3 sw=3 et ft=tex
%-------------------------------------------------------------------------------
