%-------------------------------------------------------------------------------
% cookbook_instruments
%-------------------------------------------------------------------------------
%
% \file        cookbook_instruments.tex
% \library     Documents
% \author      Chris Ahlstrom
% \date        2015-07-12
% \update      2015-07-13
% \version     $Revision$
% \license     $XPC_GPL_LICENSE$
%
%     Provides the cookbooks_instruments section of yoshimi-cookbook.tex.
%
%-------------------------------------------------------------------------------

\section{Creating Instruments}
\label{sec:cookbook_instruments}

   One of our goals in using \textsl{Yoshimi} is to support
   \textsl{General MIDI} (\textsl{GM})
   to the greatest extent possible.

   However, no banks have been created with GM in mind.  And many of the
   instruments, though given names that indicate what they are intended to
   be, fall well short of being recognizable per their name; they should be
   doable with a complex synthesizer like \textsl{Yoshimi}.
  
   It is true that there are a vast number of patches out there.  The author
   has attempted a survey of them, and the task is all but impossible.
   Still, many candidates have been identified.  Other candidates might be
   suitable with a little tweaking.

   Here are a number of categories of instruments for which we want to
   assemble an improved set of instruments.

   \begin{enumber}
      \item \textbf{Bells}
      \item \textbf{Ethnic}
      \item \textbf{Drums}
      \item \textbf{Effects}
      \item \textbf{Piano}
      \item \textbf{Leads}
      \item \textbf{Guitar}
      \item \textbf{Strings} (individual and ensemble)
      \item \textbf{Bass}
      \item \textbf{Saxophones}
   \end{enumber}

   Of course, this is too much work to do all at once.  In the next section
   we will assemble a quick-and-dirty basic GM bank with a lot of holes to
   fill in, and add the instruments we create here as each matures.

\subsection{Bells}
\label{subsec:cookbook_instruments_bells}

   The bells patches we've heard so far are nice, but a bit anemic.
   
   Good bell patches need to include ring modulation, done right.  We're not
   sure if there are any such patches extant; please send us to them if
   there are some.

   In the meantime, creating bells is a good excuse to master
   \textsl{Yoshimi}'s ring modulator.

   Start with a fresh \textsl{Yoshimi} and a cleared instrument ("Simple
   Sound").  Open the virtual keyboard using the \textbf{virKbd} button.
   Click a key and verify that you can hear a tone.  We'll use the middle C
   key (the "comma" on the PC keyboard) as a reference.

   The following steps will set up two tones, voice 1 and voice 2, and voice
   2 will use voice 1 as an external modulator.
 
   \begin{enumber}
      \item Open the ADDsynth editing window by clicking the \textbf{Edit} button
         in the bottom panel, and then clicking the ADDsynth \textbf{Edit}
         button in the edit window.
      \item Click on the \textbf{Show Voice Parameters} button.
         Note that it is \textbf{Current Voice 1}, and it should be enabled.
      \item Switch to \textbf{Current Voice} number 2 and enable it.
         Move the \textbf{FREQUENCY Detune} slider a bit, and play the "C"
         note.  It should sound the same as before, but change slowly in
         amplitude, as heard and as seen on the \textbf{VU meter}.
         Try to set the detune back to 0; this is easier if you highlight
         the tuning knob and use the left or right arrow keys.
      \item In the \textbf{MODULATOR} section of voice 2, for \textbf{Type},
         select the \textbf{RING} value.  (However, feel free to select one
         of the other modulators, to experiment, once you've mastered
         the ring modulator.)  Press the "C" key again, and notice
         that the tone is a lot louder, but not yet modulated.
      \item In \textbf{External Mod.} for voice 2, select
         \textbf{Ext.M} 1, to use the voice 1 as the internal modulator.
      \item To actually hear some modulation, we have to separate the
         frequencies of the two voices.  Click the \textbf{440Hz} check-box in
         the \textbf{FREQUENCY} section.  Press the "C" key and verify hearing
         a two-tone signal, somewhat like a phone tone.
         One can unselect the \textbf{440Hz} button.  If you go back to 
         voice 1 and select the \textbf{440Hz} button there, the same effect
         will occur.
         But unselect it again, and verify that only one apparent (loud)
         tone now sounds.
      \item Now go back to voice 2 and click the \textbf{Change} button to
         bring up the ADDsynth oscillator dialog.
      \item Move the slider to maximum for harmonic 10.  Press the "C" key
         and verify the new sound (a bit like a car horn).
         Set the sliders back to 0, and "C" will be a single tone again.
      \item Change the \textbf{Octave} values of voice 2 in its
         \textbf{FREQUENCY} section and listen to the effects.
   \end{enumber}

   Now we need to see if we can apply modulation across instruments!

\subsection{Ethnic}
\label{subsec:cookbook_instruments_ethnic}

   We've found a simple steel drum instruments, but think we might do better,
   creating one using ADDsynth and one using PADsynth.

   Instruments we have not found, and would like to synthesize, are:
   bagpipes and arabic pipes.

\subsection{Drums}
\label{subsec:cookbook_instruments_drums}

   We want a decent drum kit that attempts to fill in the gaps for a
   GM-compliant drum kit with solid sounds, and we think we've done it, with
   the help of an existing kit.

\subsection{Effects}
\label{subsec:cookbook_instruments_effects}

   We've created a nice dial-tone effect that we'll describe here.

\subsection{Piano}
\label{subsec:cookbook_instruments_piano}

   TODO.

\subsection{Leads}
\label{subsec:cookbook_instruments_leads}

   TODO.

\subsection{Guitar}
\label{subsec:cookbook_instruments_guitar}

   TODO.

\subsection{Strings} (individual and ensemble)
\label{subsec:cookbook_instruments_strings}

   TODO.

\subsection{Bass}
\label{subsec:cookbook_instruments_bass}

   TODO.

\subsection{Saxophones}
\label{subsec:cookbook_instruments_saxophones}

   TODO.

%-------------------------------------------------------------------------------
% vim: ts=3 sw=3 et ft=tex
%-------------------------------------------------------------------------------
