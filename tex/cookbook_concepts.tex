%-------------------------------------------------------------------------------
% cookbook_concepts
%-------------------------------------------------------------------------------
%
% \file        cookbook_concepts.tex
% \library     Documents
% \author      Chris Ahlstrom
% \date        2015-07-15
% \update      2015-07-15
% \version     $Revision$
% \license     $XPC_GPL_LICENSE$
%
%     Provides the concepts.
%
%-------------------------------------------------------------------------------

\section{Concepts}
\label{sec:concepts}

   This section, like its counterpart in our \textsl{Yoshimi User Manual},
   presents some useful concepts, while keeping them out of the way.

\subsection{Concepts / Terms}
\label{subsec:concepts_terms}

   This section doesn't provide comprehensive coverage of terms.  It
   covers mainly terms that puzzled the author at first.

\subsubsection{Concepts / Terms / cent}
\label{subsubsec:concepts_terms_cent}

   The \textbf{cent}
   \index{cent}
   is a logarithmic unit of measure used for musical
   intervals.  Twelve-tone equal temperament divides the octave into 12
   semitones of 100 cents each. Typically, cents are used to measure
   extremely small finite intervals, or to compare the sizes of comparable
   intervals in different tuning systems.
   The interval of one cent is much too small to be heard between
   successive notes.

   Since the detuning provided in \textsl{Yoshimi} is based primarily on
   cents (and octaves), it pays to understand cents.  If a given frequency
   \texttt{f'} is offset from another frequency \texttt{f}, the
   relationships between them in semitones are:

   \[f' = f * 2^s/12\]

   \[s = 12 log (f'/f) / log 2\]

   In cents, these relationships become:

   \[f' = f * 2^s/1200\]

   \[s = 1200 log (f'/f) / log 2\]

   These relationships hold whether \texttt{f'} is less than or greater than
   \texttt{f}.  They provide an easy way to determine how much to detune a
   frequency in \textsl{Yoshimi}.

%-------------------------------------------------------------------------------
% vim: ts=3 sw=3 et ft=tex
%-------------------------------------------------------------------------------
