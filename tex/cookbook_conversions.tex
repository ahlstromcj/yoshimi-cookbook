%-------------------------------------------------------------------------------
% cookbook_conversions
%-------------------------------------------------------------------------------
%
% \file        cookbook_conversions.tex
% \library     Documents
% \author      Chris Ahlstrom
% \date        2016-04-26
% \update      2016-04-27
% \version     $Revision$
% \license     $XPC_GPL_LICENSE$
%
%     Provides a tutorial on installing Yoshimi cookbook resources.
%
%-------------------------------------------------------------------------------

\section{Converting Tunes for Yoshimi Playback}
\label{sec:conversions}

   This section explains various methods for converting tunes to play in
   \textsl{Yoshimi}, plus some concrete methods for configuration
   \textsl{Yoshimi} and taking advantage of its newer capabilities.

   We will subdivide this section by tunes converted.

\subsection{Tune / "Before You Accuse Me"}
\label{subsec:conversions_tune_b4uacuse}

   This file describes the importing an old (circa 1996)
   Cakewalk transcription, a rendering of Eric Clapton and Robert Cray from a
   transcription that someone gave to me to try out.  It was originally
   transcribed to fit the Yahama PSS-790 \cite{pss790} 
   consumer-level keyboard, as we recall (it was years and years ago.)

   First, let us run the \texttt{b4uacuse.mid} MIDI file
   as is through the TiMidity++ \cite{timidity} software synthesizer:

   \begin{verbatim}
      $ timidity b4uacuse.mid 
      Format: 1  Tracks: 14  Divisions: 120
      Track name: PSS-790 Patchin
      Track name: Guitar 1 (E.C.)
      Track name: Guitar 2 (R.C.)
      Track name: Vocal
      Track name: Rhythm (Chords)
      Track name: Bass Guitar
      Track name: Drums
      Track name: Chris Ahlstrom
      Track name: (GEnie: KICKAHA
      Track name: "Before You
      Track name: Accuse Me" by
      Track name: Eric Clapton &
      Track name: Robert Cray
      No instrument mapped to tone bank 0, program 12 -
          this instrument will not be heard
      No instrument mapped to tone bank 0, program 87 -
          this instrument will not be heard
   \end{verbatim}

   Our \texttt{/etc/timidity/timidity.cfg}
   file specifies the \texttt{/etc/timidity/freepats.cfg} file.
   In this file, we see that bank 0 does not define the three patch numbers
   that are noted as missing above and below:  12, 62, and 87.  Patch 12 on the
   PSS-790 is
   \textsl{Marimba}, but is mapped to GM
   \textsl{Jazz Guitar} 
   (\textsl{Elec. Guitar jazz}).
   In \textsl{TiMidity++} freepats, this is patch 26.
   Patch 0 is mapped to
   patch 62, which is
   \textsl{Synth Brass 1} in GM;
   the closest in \textsl{TiMidity++} freepats is
   patch 61
   \textsl{Brass Section}.  Patch 87 on the PSS-790 is
   \textsl{Lead 8 (bass+lead)}
   which gets mapped to
   \textsl{Flugelhorn} 
   \textsl({English Horn}).
   The tune sounds reasonable, but obviously is not scored correctly.
   Let us use \textsl{midicvtpp} \cite{midicvt}
   to convert it to good (we hope) GM format:

   \begin{verbatim}
      $ midicvtpp --m2m GM_PSS-790_Multi.ini -i b4uacuse.mid -o b4uacuse-gm-1.mid
   \end{verbatim}

   or, for more output information:

   \begin{verbatim}
      $ midicvtpp --m2m GM_PSS-790_Multi.ini \
            --summarize b4uacuse.mid b4uacuse-gm-1.mid 2> summary.txt
   \end{verbatim}

   The command creates a file that has only one "error" message and
   plays pretty well in Timidity:

   \begin{verbatim}
      No instrument mapped to tone bank 0, program 62 -
          this instrument will not be heard
   \end{verbatim}

   Again, note that patch 62 is a \textsl{Synth Brass 1} patch.

   We cannot tell what the target voices of the PSS-790 were, and what they
   were converted to.  That sounds like a feature to add to the
   \textsl{midicvtpp} program!
   Done, in version 0.4.0!
   We can go forward with the GM MIDI bank we have created
   for \textsl{Yoshimi}.
   Take a look at the file \texttt{sequencer64/b4uacuse/summary.txt}
   in this project.  It is used in the patch descriptions here.

   The next step is to see how these patches map to our preliminary
   \textsl{Yoshimi} GM
   bank.  We have three ways to go for patching:

   \begin{enumber}
      \item Keep the program-change track(s) and use them to select instruments.
      \item Move each program-change to the appropriate track.
      \item Remove all program-change events and rely on a Yoshimi state file
         to set up.
   \end{enumber}

The disadvantage of keeping the program changes:

   \begin{enumber}
      \item They are currently spread out in three different tracks, as can be
         seen by opening the b4uacuse-gm-1.mid file in Sequencer64 and viewing
         the "PSS-790 Patch", "Chris Ahlstro", and "(GEnie: KICKA" tracks using
         the Event Editor.
      \item They make it difficult to remix the song for playback on various
         synthesizers other than Yoshimi.
      \item At least in \textsl{Yoshimi}, it becomes more difficult to change
         the instruments used.
      \item They take away a chance to learn how \textsl{Yoshimi} state works.
   \end{enumber}

   On the other hand, it will be interesting to add program changes to the tune
   and see how \textsl{Yoshimi} handles it these days.  Later.

   So let us take \texttt{b4uacuse-gm-1.mid}, trim out the patch tracks and
   empty tracks, recreate the layout of the song editor, and save it as
   \texttt{b4uacuse-gm-patchless.midi} (a \textsl{Sequencer64} format).
   Here are the tracks, and their putative GM
   instruments as numbered in our GM bank:

   \begin{verbatim}
      Track 0: Guitar 1 (E.C.)         #26 Acou. Guitar (steel)
      Track 1: Guitar 2 (R.C.)         #27 Elec. Guitar (jazz)
      Track 2: Bass Guitar             #34 Elec. Bass (finger)
      Track 3: Drums                   Drums
      Track 4: Vocal                   #70 English Horn
      Track 5: Rhythm (Chords)         #4 Honky-tonk Piano
   \end{verbatim}

   See \texttt{summary.txt} to see what conversions were made by
   \textsl{midicvtpp}.  We go to
   \textbf{Yoshimi / Menu / Patch Sets / Show Patch Banks...}
   and look at the banks.  We select \textsl{116. gm-basic}.
   This bank has passable (in most cases) equivalents
   to GM instruments, plus a drum kit at program number 129 (an extended
   program number).

   We will set the channels to "track + 1" (channels are counted from 1),
   except for drums, which go to channel 10.  We enable each
   \textsl{Yoshimi} Part.  We
   note the name (in quotes) of the corresponding Yoshimi patch.

   \textsl{Yoshimi} parts are taken from the \textsl{gm-basi}
   c bank ("116. gm-basic"):

\begin{verbatim}
Track 0, Ch.  1: Guitar 1 (E.C.) #26 Acoustic Guitar (steel) ("Clean Guitar 1")
Track 1, Ch.  2: Guitar 2 (R.C.) #27 Electric Guitar (jazz) ("Electric Guitar")
Track 2, Ch.  3: Bass Guitar     #34 Electric Bass (finger) ("Electric bass 1")
Track 3, Ch. 10: Drums           #129 Drum Kit ("Natural Drum Kit from DS 2")
Track 4, Ch.  5: Vocal           #70 English Horn ("Analog Brass 2")
Track 5, Ch.  6: Rhythm (Chords) #4 Honky-tonk Piano ("Synth Piano 3 fat")
\end{verbatim}

   Note that The bank must be installed in
   \texttt{\textasciitilde/.config/yoshimi-cookbook/banks} or
   in a root path setup as in
   \sectionref{subsec:cookbook_banks_root_paths}.

   After setting up each Track/Part/Channel, we played the tune, using
   ALSA, and adjusted
   the volume of some of the parts while playing it through buss 5 (where
   \textsl{Yoshimi} sets up on our system) via the command

   \begin{verbatim}
      $ sequencer64 --bus 5 b4uacuse-gm-patchless.midi
   \end{verbatim}
   
   We then saved the entire state of \textsl{Yoshimi} to
   \texttt{~/.config/yoshimi/yoshimi-b4uacuse-gm.state}
   via \textbf{Menu / State / Save ...}.  We reload this file via

   \begin{verbatim}
      $ yoshimi -a -A --state=yoshimi-b4uacuse-gm.state
   \end{verbatim}
   
to verify that it still works.  We ended up switching to JACK playback for
recording the audio:

   \begin{verbatim}
      $ sequencer64 -J b4uacuse-gm-patchless.midi
      $ yoshimi -j -J --state=yoshimi-b4uacuse-gm.state
   \end{verbatim}

   and \textsl{avconv} (\textsl{ffmpeg}) for converting it to an MP3 file that
   one can listen to without loading up all this software.

%-------------------------------------------------------------------------------
% vim: ts=3 sw=3 et ft=tex
%-------------------------------------------------------------------------------
