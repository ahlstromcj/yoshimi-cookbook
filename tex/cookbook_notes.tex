%-------------------------------------------------------------------------------
% cookbook_notes
%-------------------------------------------------------------------------------
%
% \file        cookbook_notes.tex
% \library     Documents
% \author      Chris Ahlstrom
% \date        2016-10-29
% \update      2016-10-29
% \version     $Revision$
% \license     $XPC_GPL_LICENSE$
%
%     Provides a tutorial on notesing Yoshimi cookbook resources.
%
%-------------------------------------------------------------------------------

\section{Usage Notes}
\label{sec:notes}

   This section contains some notes about usage and settings we've collected
   that don't fit anywhere else.  These notes come from 
   \url{http://www.freelists.org/list/yoshimi}.
   To post to the list, email to: \url{yoshimi@freelists.org}
   The archive of the old SourceForge mailing list is found
   at: \url{https://www.freelists.org/archive/yoshimi}.

\subsection{Notes / Instrument Design}
\label{subsec:notes_xxxxx}

   From user "Ichthyostega":

   \begin{quotation}
      I find it difficult to balance or voice an instrument design
      such that it "runs well" when used in composition over various octaves.
      With some designs, the descant dominates, while in other cases the
      bass dominates. When I play a chord over several octaves, or have
      several lines at the same time, they show a different weight
      and are not well balanced out. Of course, I can somewhat fix
      that by giving the individual notes different velocities or
      use several MIDI channels with different volume settings.
      But I rather consider this the job of instrument design:
      it should come out well balanced out of the box.
   \end{quotation}

   Will answers:

   \begin{quotation}
      First off, the way the sound engines work involves a degree of modelling
      whereby sound energy tends to decrease with increasing frequency. This
      was a deliberate choice made by Paul when he designed them, intending to
      more-or-less emulate real instruments. It's been a minor nuisance to me
      occasionally, but to a degree you can get round it by fiddling the
      \textbf{Resonance} graph (not SubSynth).
      You can also set filters to \textbf{LPF1}
      with \textbf{Q} to zero and \textbf{C.freq} to max.

      Any waveform with little harmonic content will tend to sound quieter at
      lower frequencies, while the amplitude is actually pretty constant.
      \textsl{Simple Sound} is a classic example of that.

      Fades across a instruments in a kit has been mentioned before. I had a
      look at it some time ago, and it is actually an
      \textsl{extremely} difficult
      problem. If there were only two possible kit items it would 'just' be
      hard -- across potentially 16 is in the nightmare realm!
   \end{quotation}

%-------------------------------------------------------------------------------
% vim: ts=3 sw=3 et ft=tex
%-------------------------------------------------------------------------------
