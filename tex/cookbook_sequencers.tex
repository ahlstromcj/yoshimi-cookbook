%-------------------------------------------------------------------------------
% cookbook_sequencers
%-------------------------------------------------------------------------------
%
% \file        cookbook_sequencers.tex
% \library     Documents
% \author      Chris Ahlstrom
% \date        2016-03-12
% \update      2016-03-12
% \version     $Revision$
% \license     $XPC_GPL_LICENSE$
%
%     Provides the discusson of using sequencers with Yoshimi.
%
%-------------------------------------------------------------------------------

\section{Sequencers}
\label{sec:sequencers}

   This section discusses some basic usage of a few seqeuencers
   with \textsl{Yoshimi}.
   Some sequencers support ALSA MIDI, some support JACK MIDI, and some
   support both.  This section provides some use cases and pointers that cover
   a range of sequencers.

   Note that this section depends on having one's Banks/Roots directories set
   up properly.  It might be useful to make sure that there is a default
   "presets" directory present.

\subsection{Sequencers / Sequencer64}
\label{subsec:sequencers_seq64}

   \textsl{Sequencer64} is our fork/reboot of the \textsl{Seq24} project.
   It currently fixes some bugs in \textsl{Seq24}, adds a few useful features,
   and refactors the code.  Much of the advice in this section will apply to
   \textsl{Seq24}, with some minor differences, such as the locations and names
   of the configuration files.

   Note that the two \textsl{Sequencer64} configuration files are:

   \begin{verbatim}
      ~/.config/sequencer64/sequencer64.rc
      ~/.config/sequencer64/sequencer64.usr
   \end{verbatim}

   The simplest thing to do for this cookbook is
   
   \begin{itemize}
      \item Make sure these files do not exist; save the old versions somewhere
      and then delete the two file.
      \item Then start \textsl{Sequencer64}.
      \item Then immediately exit it.
   \end{itemize}
      
   New versions of those files will appear, and we can start with a clean
   slate.  We will edit those files, as that is a bit more foolproof than using
   command-line options.

   The "rc" is the most important file.  There are configuration items in the
   "usr" ("user") file that can make \textsl{Sequencer} prettier, and a couple
   options that can affect playback.  We'll point out the necessary values when
   needed.

\subsubsection{Sequencers / Sequencer64 / ALSA}
\label{subsubsec:sequencers_seq64_alsa}

   Skip this section if all that matters for one's setup is JACK support.
   Otherwise, make sure JACK is not running.
   Next, remove the "rc" and "usr" files from the configuration directory, as
   noted in the previous section.
   Then open up a console window (terminal) and run \textsl{Yoshimi} in ALSA
   audio and ALSA MIDI modes:

   \begin{verbatim}
      $ yoshimi -a -A
   \end{verbatim}

   Now, run \textsl{Sequencer64}, and check the new
   versions of those files, as noted in the previous section.
   On our computer, with Timidity installed, we see the following ALSA entries
   in the \texttt{sequencer64.rc} file:

   \begin{verbatim}
		# Output buss name: [0] 14:0 Midi Through Port-0
		0 0  # buss number, clock status
		# Output buss name: [1] 128:0 TiMidity port 0
		1 0  # buss number, clock status
		# Output buss name: [2] 128:1 TiMidity port 1
		2 0  # buss number, clock status
		# Output buss name: [3] 128:2 TiMidity port 2
		3 0  # buss number, clock status
		# Output buss name: [4] 128:3 TiMidity port 3
		4 0  # buss number, clock status
		# Output buss name: [5] 129:0 input
		5 0  # buss number, clock status
   \end{verbatim}

   The last entry in that list is "129:0 input", which corresponds to
   \textsl{Yoshimi}.  Note that the \textsl{Sequencer64} application will
   find the actual name of the device (e.g. "129:0 Yoshimi Port 0").
   
   While we're in the "rc" file, let's make sure of the following settings:

   \begin{verbatim}
      1     # show sequence numbers (1 = true / 0 = false); ignored in legacy
		[manual-alsa-ports]
		0     # flag for manual ALSA ports
   \end{verbatim}

   The first setting enables showing the sequence number in empty slots in the
   main window.  (This setting really belongs in the "user" file!)
   The second setting disables manual setup of the ALSA MIDI ports.
   \textsl{Sequencer64} will determine the ports that exist on the system.

   In the "user" file, we want to set the buss number to 5, to match our
   current ALSA setup.  It may differ on your system.  If it is 0, you don't
   have to do the next step.

   Open the "user" file and make sure the following setting matches the
   location of \textsl{Yoshimi}'s MIDI input port on your system (it is 5 on
   ours):

   \begin{verbatim}
   5       # midi_buss_override
   \end{verbatim}

   Now open a MIDI file, verify that the correct buss value is shown, and that
   the file plays and is audible.  If this is the case, one is done.  If not,
   please read the \textsl{Sequencer64} user's manual
   (\cite{sequencer64doc}).

   Finally, we are ready to add a loop.  In \textsl{Sequencer64}, right click
   on a pattern slot and select \textbf{New}.  On the bottom of the new pattern
   window, select the left-most icon button, which shows a blue box pointing to
   a MIDI port (tooltip: "Sequence dumps data to MIDI bus.").
   Press the Space bar while in the main window or the pattern window
   to start the loop, and add notes.

   \begin{quotation}
      \textbf{Peculiarities of Sequencer64}.
      First, if the pattern is empty the
      progress bar will not move.  Second, to add a note, one must press
      the right button (the pointer changes to a pencil) and, while holding
      it, press the left mouse button.  Or click in the pattern editor, press
      "p" to select the pencil mode, then click/drag to add notes.  Press "x"
      to "eXit" from that mode.  Also note that notes are drawn only with the
      length selected by the "notes" button near the top of the pattern window.
      Again, see the Sequencer64 user manual for the gory details.
   \end{quotation}

%-------------------------------------------------------------------------------
% vim: ts=3 sw=3 et ft=tex
%-------------------------------------------------------------------------------
